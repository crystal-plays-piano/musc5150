\section*{Chanson - Nadia Boulanger}


\begin{minipage}[t][0.7\textheight][t]{\textwidth}

  \begin{multicols}{2}
    \textbf{Chanson}

    \hfill{}

Les lilas sont en folie,\\
Cache cache\\
Et les roses sont jolies,\\
Cachez-vous.

    \hfill{}

Tirez les rideaux, tirez les rideaux !\\
Et sous les vertes feuilles\\
Cachez-vous !

    \hfill{}

Ah ah! Ah ah! Ah ah!

    \hfill{}

Lilas et rosiers\\
la belle,\\
la plus belle, c'est toi !

    \hfill{}

Beaux seigneurs et dames belles,\\
aime, aime,\\
dans vos atours de dentelles,\\
Aimez-vous.

    \hfill{}

Tirez les rideaux !\\
Qui voudra de mon âme?\\
Aimez-vous !

    \hfill{}

Ah ah! Ah ah! Ah ah!

    \hfill{}

Amours et baisers, la belle\\
Ah ah! Ah ah!\\
la plus belle c'est toi !

\newcolumn{}

\textbf{Song}

    \hfill{}

The lilacs are inflamed,\\
Hide-and-seek,\\
And the roses are pretty,\\
Hide yourself.

    \hfill{}

Draw the curtains, draw the curtains!\\
And beneath the green leaves\\
Hide yourself!

    \hfill{}

Ah ah! Ah ah! Ah ah!

    \hfill{}

Lilacs and rose-bushes Ah ah!\\
The fair one, Ah ah! Ah ah!\\
The fairest one is you!

    \hfill{}

Handsome lords and beautiful ladies,\\
Love, love,\\
In your silken finery,\\
Love.

    \hfill{}

Draw the curtains, draw the curtains!\\
Who would like my soul?\\
Love!

    \hfill{}

Ah ah! Ah ah! Ah ah!

    \hfill{}

Love! Ah ah! Ah ah! Ah ah!\\
Love and kisses, ah the fair one,\\
Ah ah! the fairest one is you!

\end{multicols}

\end{minipage}

\begin{minipage}[t][0.2\textheight][t]{\textwidth}
  Nadia Boulanger lived a long, long life - from 1887 to 1979 - during
  which she was primarily known as a teacher of composition. Her numerous
  students include Elliot Carter, Aaron Copland, Philip Glass, Quincy Jones,
  and Astor Piazzola; that is, her teachings influenced almost an entire
  century of composers, and her distinct musical voice is reflected in the
  work of hundreds. It's rather unfortunate, then, that Nadia Boulanger stopped
  composing soon after her sister's passing in 1918. This piece is on the livelier
  side, with bountiful laughing and a bright, cheerful melody. You can hear the
  pure, bashful joy of having a crush on someone, not knowing if they admire you
  back. The poem is written by a Georges Delaquy (1880-1970), who was a frequent
  songwriter for both Nadia Boulanger and her sister Lili Boulanger.

\end{minipage}
