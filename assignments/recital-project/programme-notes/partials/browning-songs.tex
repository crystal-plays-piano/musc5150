\begin{minipage}[t][0.5\textheight][t]{\textwidth}
\section*{\textbf{3 Browning Songs} - Amy Beach}


\begin{multicols}{3}
  \textbf{The Year's at the Spring}

  \hfill{}

  The year’s at the spring,\\
  And day’s at the morn;\\
  Morning’s at seven;\\
  The hill-side’s dew-pearl’d;\\
  The lark’s on the wing;\\
  The snail’s on the thorn;\\
  God’s in His heaven–\\
  All’s right with the world!

  \newcolumn
  % \newcolumn{}

  \textbf{Ah, Love, but a day}

\hfill{}

  Ah, Love, but a day,\\
  And the world has changed!\\
  The sun’s away,\\
  And the bird estranged;\\
  The wind has dropped,\\
  And the sky’s deranged;\\
  Summer has stopped.

  \hfill{}

  Look in my eyes!\\
  Wilt thou change too?\\
  Should I fear surprise?\\
  Shall I find aught new\\
  In the old and dear,\\
  In the good and true,\\
  With the changing year?

  ---

  % \hfill{}


  \textit{Thou art a man,\\
  But I am thy love.\\
  For the lake, its swan;\\
  For the dell, its dove;\\
  And for thee — (oh, haste!)\\
  Me, to bend above,\\
  Me, to hold embraced.}




  \hfill{}

  \textit{(Beach set only the first two stanzas in her song.)}

  \newcolumn{}

  \textbf{I Send My Heart Up To Thee}

\hfill{}

  I send my heart up to thee, all my heart/\\
  In this my singing,/\\
  For the stars help me, and the sea, and the sea bears part;/\\
  The very night is clinging/\\
  Closer to Venice’ streets to leave on space/\\
  Above me, whence thy face/\\
  May light my joyous heart to thee, to thee its dwelling place.



\end{multicols}
\end{minipage}

  \begin{minipage}[b][0.25\textheight][t]{\textwidth}
      \vspace{1in}

      \begin{quotation}
      \centering{}
        ``It has happened more than once that a composition comes to
        me, ready made as it were, between the demands of other work.
        \textit{The Year’s At The Spring} was “born” the same way."
      \end{quotation}

      \hfill{}

      So describes composer Amy Beach (1867 - 1944) the process of her realising
      of these Robert Browning poems. According to her story, Beach
      had put off the composition of \textit{The Year's at the Spring}
      until just before the expected date of first performance. Under
      pressure from the upcoming deadline, Beach wrote the song in
      one train ride from New York back to Boston. Indeed, by her own
      admission, much of Beach's composition is impulsive - ``[the music]
      jumped at me and struck me, most forcibly!" - and it is reflected
      on the openness and directness of her music. The emotional highs
      and lows are treated with utmost dramatic intensity and are
      especially well represented in this set of songs set to texts by
      Robert Browning. The set is dedicated to the Boston Browning Society.



  \end{minipage}
