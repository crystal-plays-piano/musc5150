\section*{from \textit{Liederkreis Op.25} - Robert Schumann }


\begin{minipage}[t][0.7\textheight][t]{\textwidth}

  \begin{multicols}{2}
    \textbf{No - 9,  Mit Myrten und Rosen}

\hfill{}

Mit Myrthen und Rosen, lieblich und hold,\\
Mit duft’gen Zypressen und Flittergold,\\
Möcht’ ich zieren dies Buch wie ’nen Totenschrein,\\
Und sargen meine Lieder hinein.

\hfill{}

O könnt’ ich die Liebe sargen hinzu!\\
Auf dem Grabe der Liebe wächst Blümlein der Ruh’,\\
Da blüht es hervor, da pflückt man es ab,—\\
Doch mir blüht’s nur, wenn ich selber im Grab.

\hfill{}

Hier sind nun die Lieder, die einst so wild,\\
Wie ein Lavastrom, der dem Ätna entquillt,\\
Hervorgestürzt aus dem tiefsten Gemüt,\\
Und rings viel blitzende Funken versprüht!

\hfill{}

Nun liegen sie stumm und totengleich,\\
Nun starren sie kalt und nebelbleich,\\
Doch aufs neu’ die alte Glut sie belebt,\\
Wenn der Liebe Geist einst über sie schwebt.

\hfill{}

Und es wird mir im Herzen viel Ahnung laut:\\
Der Liebe Geist einst über sie taut;\\
Einst kommt dies Buch in deine Hand,\\
Du süsses Lieb im fernen Land.

\hfill{}

Dann löst sich des Liedes Zauberbann,\\
Die blassen Buchstaben schaun dich an,\\
Sie schauen dir flehend ins schöne Aug’,\\
Und flüstern mit Wehmut und Liebeshauch.


\newcolumn{}

\textbf{No 9 -  With myrtles and roses }

\hfill{}

With myrtles and roses, sweet and fair,\\
With fragrant cypress and golden tinsel,\\
I should like to adorn this book like a coffin\\
And bury my songs inside.

\hfill{}

Could I but bury my love here too!\\
On Love’s grave grows the flower of peace,\\
There it blossoms, there is plucked,\\
But only when I’m buried will it bloom for me.

\hfill{}

Here now are the songs which once cascaded,\\
Like a stream of lava pouring from Etna,\\
So wildly from the depths of my soul,\\
And scattered glittering sparks all around!

\hfill{}

Now they lie mute, as though they were dead,\\
Now they stare coldly, as pale as mist,\\
But the old glow shall kindle them once more,\\
When the spirit of Love floats over them.

\hfill{}

And a thought speaks loud within my heart,\\
That the spirit of Love will one day thaw them;\\
One day this book will fall into your hands,\\
My dearest love, in a distant land.

\hfill{}

Then shall song’s magic spell break free,\\
And the pallid letters shall gaze at you,\\
Gaze imploringly into your beautiful eyes,\\
And whisper with sadness and the breath of love.

\end{multicols}

\end{minipage}

\begin{minipage}[t][0.2\textheight][t]{\textwidth}
  Almost all of Robert Schumann's lieder - German song - production
  was during the miraculous period lovingly referred to as his ``Liederjahr"
  (year of song), where, directly after his marriage to Clara Wieck - another
  well respected composer and pianist - in 1840, he spent the next year dedicated to
  writing almost exclusively songs. These songs span the breadth of his musical
  style, from bombastic showpieces like \textit{ Der Contrabandiste } to
  more intimate works like \textit{Im wunderschönen Monat Mai} and this piece,
  \textit{Mit Myrten und Rosen}, or ``With myrtles and roses". The poem - by
  German poet Heinrich Heine (1797-1856) - is a languid painting of a
  love buried so deep that it will never reach the intended recipient.
  Hopefully, though the love will reach the audience. After 1841, his song output
  diminished. Robert Schumann lived
  from 1810 to 1856, though he spent the last two years of his life in a mental
  institution, and his musical output stopped in 1854. Among his last pieces
  is, curiously, a return to song - though this time for solo piano - in the
  enigmatic \textit{Gesänge der Frühe}.
\end{minipage}
