\section*{from \textit{12 Romances}, Op. 21 - Sergei Rachmaninoff}

\begin{minipage}[t][0.7\textheight][t]{\textwidth}

  \begin{multicols}{2}

    \textbf{No. 5 - Siren', 'Lilacs'}

\hfill{}

    Poutru, na zare,\\
Po rasistoj trave,\\
Ya pajdu svezhym utrom dyshat’;\\
I v dushystuyu ten’,\\
Gde tesnitsya siren’,\\
Ya pojdu svoyo shchast’ye iskat’...

 \hfill{}

V zhizni shchast’ye odno\\
Mne najti suzhdeno,\\
I to shchast’ye v sireni zhyvyot;\\
Na zelyonykh vetvyakh,\\
Na dushistykh kistyakh\\
Moyo bednoe shchast’ye tsvetyot...

  \newcolumn{}

  \textbf{No. 5 - Lilacs}

  \hfill{}

  In the morning, at dawn,\\
Through the dew-clad grass,\\
I shall walk, breathing in the freshness of morning;\\
And to the fragrant shade,\\
Where lilacs cluster,\\
I shall go in search of happiness…

\hfill{}

In life there is but one happiness\\
That I am fated to find,\\
And that happiness dwells in the lilacs;\\
On their green branches,\\
In their fragrant clusters\\
My poor happiness blooms…


\end{multicols}

\end{minipage}

\begin{minipage}[t][0.2\textheight][t]{\textwidth}

  Sergei Rachmaninoff, renowned Russian Composer, lived from 1873 to 1943, and,
  like many Russian artists of the time, left the country following political
  turmoil in 1917. Notably, Rachmaninoff's last published songs - the \textit{12
  Romances} - were composed in 1917. These songs, then, reflect Rachmaninoff's
  earlier writing - intimate, flowing, lyrical, and harmonically colourful.
  This song in particular was well loved my the composer and rearranged for
  solo piano later in life. The text is taken from Russian poet Ekaterina
  Andreyena Beketova, who lived from 1855 to 1892.

\end{minipage}
