\documentclass{article}

% - imports
\usepackage{enumitem}
% \usepackage[backend=biber]{biblatex-chicago}
% \addbibresource{bibliography.bib}
\usepackage{url}
\usepackage{parskip}
\usepackage{indentfirst}
\usepackage[utf8]{inputenc}
\usepackage{parallel}

\title{\vspace*{-72pt} French 1 }
\author{Crystal Mandal}
\date{Last Edited \today}

\begin{document}

  \maketitle

  \section*{Question 1}
    \subsection*{Chapter 1 - \textit{Why This Guide?}}
      \begin{enumerate}[label=(\alph*)]
        \item{While I am in general less familiar with French musical
          traditions than I am with American and German traditions, I
          am familiar with the music of Debussy, Fauré, and Poulenc. I
          am immediately struck, however, with the focus on this era of
          French music, which I would consider modern\*. Some of my
          Favourite music is the music of Josquin des Prez. Is this music
          antiquated in the same way to more popular and modern French
          Song traditions in the same way that some of the music of Bach
          is to Schubert?}
        \item{In a more informal tone: \textit{Wow}, I do \textit{NOT}
          know any French poets. I don't recognise any of the ones
          mentioned in this Chapter.}
        \item{What strikes me most is the implication that there is more
          to French rhyme patterns than simply sound. I am aware that
          French is a language where much of it is simply not pronounced
          (or, more accurately, the pronunciation is predicated by a
          phonetics that overloads different sounds in many different
          ways that I am not familiar with) but had never considered
          the rhythmic implications of that non-pronunciation.}
        \item{I'm very excited to learn new things in our study of
          French song. Franck and Vierne are among my Favourite composers:
          are either of them particularly well known for their Songs?
          Also, when talking about German Songs, we've been calling
          many of them Lieder. Is there a special term for French Songs
          as well? I've heard the terms chant and chanson, but I'm not
          sure of their prevalence as compared to Lieder.}
      \end{enumerate}

    \subsection*{Chapter 2 - \textit{The Basics of French Line}}
      \begin{enumerate}[label=(\alph*)]
        \item{I am, though likely less well read than the rest of the class,
          at least generally familiar with the English language and some of
          its poetic and rhythmic quirks, so the comparisons between the
          Alexandrine and the sonnet were very helpful.}
        \item{I am very unfamiliar with French song, literature, language,
          and pronunciation, so this guide should be very useful to me. For
          starters, I had no idea about the Alexandrine as a poetic form;
          I've heard of it, yes, but until now I hadn't taken the time to
          even learn the basic structure.}
        \item{I think an early internalisation that, while English shares
          quite a bit of vocabulary from French - every time I see a new
          French loanword I'm surprised I already know it - much of the
          English sentence and syllable structure is significantly closer
          to German/Germanic Language (though this may just be an effect of
          my familiarity with German pronunciation) is likely the most
          important concept to consider in the early process of learning 
          about French poetry and song.}
        \item{I don't think I know enough about this topic yet to have 
          any insightful Questions or at least any confusion that couldn't 
          be solved by reading an encyclopedia.}
      \end{enumerate}

    \subsection*{Appendix 1 - \textit{Le Colibri}}
      \begin{enumerate}[label=(\alph*)]
        \item{Nothing. I've never seen this poem, never heard of this
          Author, and never realised just how little French I know how
          to pronounce or read.}
        \item{The inverses of everything mentioned above (the poem,
          author, and pronuncitions) are all new to me, as well as
          the very detailled analysis of French rhyme and meter, much
          of which took me multiple readings to really comprehend.}
        \item{For me, the obsessive analysis of the French phonetics
          involved in constructing this rhyme scheme is the most
          important part. I'm not sure I feel comfortable doing this
          sort of line-by-line analysis of the language just yet, but
          It's very helpful to see it broken down so neatly.}
        \item{Is there a particular collection/anthology of French poetry
          that I could read that would prepare me better for this class?
          Or, just generally, is there a mixed collection of French
          poetry (+ translations, because I, unfortunately, do not
          speak French) that you would recommend?}
      \end{enumerate}

    \subsection*{Appendix 3 - \textit{Chronology of French Verse}}
      \begin{enumerate}[label=(\alph*)]
        \item{I do recognise some of these Authors! The few I recognise
          are: Charles d'Orléans, La Fontaine, Lamartine,
          (Victor, I assume) Hugo, Gautier, (Is it Robert?) Bonnières.
          Unfortunately, I don't believe I have read any of these
          Authors, which, to me, means I am less familiar with them
          than if I concretely knew I hadn't read any work of theirs.}
        \item{I'm not sure I know enough for any of this timeline to be
          especially useful, unfortunately.}
        \item{It is helpful that this timeline becomes progressively more
          granular after 1800. The granularity clarifies, for me, the
          intended era of study of this particular guide.}
        \item{Too many to count, but, again, nothing insightful or
          more involved than can be answered with an encyclopedia.}
      \end{enumerate}
\clearpage
  \section*{Question 2}
    \textit{Crystal's Comment (for fun, unrelated to the assignment):}
    Oh my \textit{GOSH} I guess I shouldn't be surprised but these are
    so delightfully \textit{French} - I know, shocker. The piano textures
    are so intricate, the harmonies so contrapuntal and clearly organic
    in a way I much more closely associate with French music. I'm a big
    fan of the way the piano part goes into three staves in the Koechlin
    setting.
    \begin{enumerate}
      \item{I never thought of French as a particularly vowel-heavy
        language. The most French sounds I can think of are the - pardon
        my very inaccurate vernacular here - guttural ``H" and ``R"
        sounds and the many varieties of ``S" and ``Z" sounds. Here,
        though, unlike the German song we have been studying, I can
        recognise very few consonants (and thus, to my untrained ear,
        syllables) to ground myself and hear the text. Thus, all of
        my thoughts come more from a close reading of the score, and
        less from any aural response. I personally thing that the
        Chausson setting is more interesting in terms of text-painting:
        the trill-like figurations in the piano part and the later
        runs and arpeggios call to other contemporaneous depictions of
        small flight as depicted on piano (though, the two pieces I
        am primarily thinking of are both titled ``Papillons", or
        ``Butterflies", which, while I think is close enough, is not
        quite the same), and, most interestingly, the piano part takes
        on the role of the hymn or chorale in the end when the poor
        hummingbird dies, as if turning the last stanza into a prayer
        or funeral for the hummingbird (note the pianistic similarities
        to Alkan's ``Priere" and the Funeral Dirge from his Grand Piano
        Sonata, as well as the use of chorale themes in Liszt's Ballades
        and Prayers). Finally, a landing on the sweetest Perfect Authentic
        Cadence describing the fond memory of a first kiss. No words:
        just pianistically and narratively beautiful.}
      \item{I think I accidentally answered both questions in the first
        answer, so I will restate that I think the Chausson setting is
        more interesting - more successful - and that the piano texture
        is much more narrative than one would expect from a barely-melodic
        accompaniment part.}
    \end{enumerate}
\clearpage{}
  \section*{Question 3}
    \begin{enumerate}
      \item{It is almost commical just how ``Spanish" most of these
        songs sound, almost entirely - from my instrumental perspective -
        from the treatment of the piano part as guitar stand-in.
        Saint-Saëns' ``Guitar" emulation is likely the most conventionally
        divorced from Spain as a guitar-sound: he starts the piece with
        wide arpeggios centered in and around the chords of E, B, A, D,
        and B-Flat. These chords, seemingly only loosely related, are
        in fact \textit{very} comfortable, idiomatic, and easy to play
        on a standard open E guitar tuning. Similarly, Lalo and Bizet use
        explicit flamenco rhythms and again are centered in the keys of
        E-Flat and A (open E-flat being a common tuning and A just being
        a standard key for the dance ``Bulerias"), further referencing
        explicitly a Spanish Guitar tradition. Unfortunately, Massenet's
        setting fails in this regard (that is, in my opinion). The
        bouncing ``oom-pah" rhythm is much more common on the piano,
        and the duple meter is less common in the Spanish Guitar
        tradition, though the rolled chords at the end are charming and
        ``strummed" like a guitar.}
      \item{With careful score study I am able to recognise some of the
        text and hear the rhythmic structures, but they are, for me,
        unfortunately hidden by some gorgeously addictive melodies and
        delightful piano textures. I think I will need a lot more
        practice with French poetry before I can produce any reasonably
        insightful discourse on the actual text of the song.}
      \item{I cannot pick between the Lalo and Bizet settings, but I will
        contend that they are so enjoyable to me for different reasons.
        I think that, even though the Bizet's texture is a little less
        ``guitar-like", the explicit harmonic references to
        \textit{Bulerias} and the melismatic (almost improvisational)
        flourishes are so delightfully ``flamenco" that it is the more
        academically successful piece; I simply think that the brighter
        Lalo setting is more beautiful, and it reminds me of the second
        dance from Liszt's \textit{Rhapsodie Espagnole}.}
    \end{enumerate}
\clearpage{}
  \section*{Question 4}
  I wasn't sure how to structure this response, especially because
  it is framed as a ``present to the class" response, so I put some
  notes here for what I was looking at/thinking of.
  \begin{itemize}
    \item{Bio} \begin{itemize}
        \item{Name: Jules Émile Frédéric Massenet}
        \item{Years: 12 May 1842 - 13 August 1912}
        \item{Location: France}
    \end{itemize}
    \item{famous opera composer (more than 30 operas)} 
    \item{Education} \begin{itemize}
      \item{Lycée Saint-Louis (non-academic studies) }
      \item{Paris Conservatoire (from early 1850s, musical studies)}
    \end{itemize}
    \item{Musical Lineage} \begin{itemize}
      \item{Met Wagner}
      \item{Admired Berlioz}
      \item{Met Franz Liszt in Rome}
      \item{Teacher Lineage:}
        \begin{itemize}
          \item{taught Charles Koechlin}
          \item{Massenet}
          \item{Ambroise Thomas (Taught at Paris Conservatoire)\\ 
            anti- Franck,Fauré}
          \item{Friedrich Kalkbrenner (teacher of Marie Pleyel)\\
            indirectly influenced/taught Gottschalk, Saint-Saëns}
        \end{itemize}
    \end{itemize}
    \item{Famous Works} \begin{itemize}
      \item{Operas} \begin{itemize}
        \item{Manon}
        \item{Werther: based on Goethe!}
        \item{Thaïs (with the famous meditation)}
        \item{Don Quixote, Cinderella}
      \end{itemize}
    \end{itemize}
    \clearpage{}
    \item{Musical Impressions} \begin{itemize}
      \item{Heavily Lyrical}
      \item{French Elegance (in lineage with Couperin, Gounod)}
      \item{In the kindest way possible, a little boring}
      \item{A reluctance toward interesting piano parts}
    \end{itemize}
  \end{itemize}
  I found some lovely piano music of his as well, but couldn't find
  anything that was commonly played. Much of the songs I've heard and
  piano work I've found seems to be more straightforward salon music
  than I had expected, especially because I was, prior to this assignment
  familiar with \textit{Thaïs} and \textit{Werther}.

  % 
\vspace*{-72pt}

\begin{Large}
  Song Sheet: F. Schubert's \textit{Der Wanderer} 
\end{Large}


% \begin{enumerate}[label=\Alph*.)]
%   \item{The Chosen Song is: Franz Schubert's \textit{Der Wanderer}, Op. 65, No., D.649 }
% \end{enumerate}

\begin{itemize}[label=]
  \item{\textbf{Title: \textit{Der Wanderer}, Op. 65, No. 2, D. 649}}
  \item{\textbf{Composed: February 1819}}
  \item{\textbf{Composer: Franz Schubert (1797-1828)}}
  \item{\textbf{Poet: Georg Philipp Schmidt ('von Lübeck', 1766-1849) formerly attributed to Zacharias Werner }}\autocite{imslp-649}
\end{itemize}



\section*{General}
This song is one of four composed by Schubert on the same poem. 
The other three are: D.649, D.795, and D.8D.649, D.795, and D.870 \autocite{tapatalk} ; 
of note, D.649 is the most well known (at least, amongst pianists) of 
the bunch, and is - quite famously - the primary subject of the great 
\textit{Fantasy in C Major} (or ``Wanderer Fantasy") of Schubert.

Approximate Performance Time: a little under 4 minutes.

\section*{Melody}

\begin{tabular}{ c | c }
  Melodic Contour    & mostly stepwise or by thirds \\ 
  Tessitura          & spans a Major 9th, from C4 to D5\\
  Vocal Articulation & I'm not sure what this means\\
  Text Illustration  & N/A
\end{tabular}



\section*{Harmony}

\begin{tabular}{ c | c }
  Texture           & Chorale Texture \\ 
  Tonality          & Minimal Modulation \\
                    & Some Chromatic Passages\\
  Text Illustration & N/A\\
\end{tabular}



\section*{Rhythm}

\begin{tabular}{ c | c }
  Rhythmic Pattern & Flowing Chorale in 8th notes \\
                   & few ``marching" interjections\\ 
  Tempo & Andante\\
\end{tabular}



\section*{The Piano Component}

\begin{tabular}{ c | c }
  Preludes/Interludes/Postludes & few solo piano passages\\ 
  Tonality       & Some chromatic passages \\
                 & heavy insistence on a melodic G\#\\
  Use of Motives & there are many references to \\
                 & the opening piano gesture\\
\end{tabular}



\section*{Poem/Text}

\begin{large}
  \textbf{\textit{Der Wanderer} - Georg Philipp Schmidt 'von Lübeck' }
\end{large}

\vspace{10pt}

\begin{Parallel}[v]{0.48\textwidth}{0.48\textwidth}
  \ParallelLText{
    Wie deutlich des Mondes Licht \\
    Zu mir spricht, \\
    Mich beseelend zu der Reise:\\
    "Folge treu dem alten Gleise,\\
    Wähle keine Heimath nicht.\\
    Ew'ge Plage\\
    Bringen sonst die schweren Tage.\\
    Fort zu andern\\
    Sollst du wechseln, sollst du wandern,\\
    Leicht entfliehend jeder Klage."\\

    Sanfte Ebb' und hohe Fluth,\\
    Tief im Muth,\\
    Wandr' ich so im [Dunkel]1 weiter,\\
    Steige muthig, singe heiter,\\
    Und die Welt erscheint mir gut.\\
    Alles reine\\
    Seh' ich mild im Wiederscheine,\\
    Nichts verworren\\
    In des Tages Gluth verdorren:\\
    Froh umgeben, doch alleine.}\

  \ParallelRText{
    How clearly the moon's light \\
    Speaks to me, \\
    Inspiring me to journey; \\
    "Follow truly the ancient path, \\
    Choose no homeland whatsoever. \\
    Otherwise the heavy days bring \\
    Endless troubles ; \\
    Away, to the other \\
    Should you change, should you wander, \\
    Lightly shedding every woe." \\

    Gentle ebb and lofty flood, \\
    Deep in courage, \\
    I wander farther in darkness, \\
    I climb bravely, singing cheerfully, \\
    And the world seems good to me. \\
    All pureness \\
    See I softly in the twilight, \\
    Without confusion \\
    Fading in the day's afterglow: \\
    Surrounded by joy, but alone.} 
\end{Parallel}


\section*{Poet}


I couldn't find much information of the Poet, simply that 
he is known for this poem alone.

\section*{Choice of Text} 
Schubert has multiple songs and other works arranging this
text. Each of his settings occupies a unique soundworld and 
emotional context. This setting, compared to his more/ dramatic 
and popular - at least amongst pianists - D.493, is more calm 
and reticent, lending to a figure of a weary, wise, and well 
travelled \textit{Wanderer}.


\section*{Prosody}
I'm not familiar with much German diction, so it seems very 
normal/characteristically German in pronunciation and stress 
to me.

  % 
\vspace*{-72pt}

\begin{Large}
  Song Sheet: R. Schumann's \textit{Der Contrabandista} 
\end{Large}


% \begin{enumerate}[label=\Alph*.)]
%   \item{The Chosen Song is: Franz Schubert's \textit{Der Wanderer}, Op. 65, No., D.649 }
% \end{enumerate}

\begin{itemize}[label=]
  \item{\textbf{Title: \textit{Der Contrabandiste}, Op. 74 }} 
  \item{\textbf{Composer: Robert Schumann (1810-1856)}}
  \item{\textbf{Poet: Emanuel von Geibel (1815 - 1884) }}\autocite{imslp-74}
\end{itemize}



\section*{General}

This song is from the appendix to R. Schumann's 
\textit{Spanisches Liederspeiel}. According to 
Hyperion records, it was removed from the original 
set due to its relative unimportance to the storyline 
of the overall set.



Approximate Performance Time: a little over 1 minute

\section*{Melody}

\begin{tabular}{ c | c }
  Melodic Contour    & very chordal, lots of jumps\\ 
  Tessitura          & almost 2 octaves, from A2 to G4\\
  Vocal Articulation & I'm not sure what this means\\
  Text Illustration  & N/A
\end{tabular}



\section*{Harmony}

\begin{tabular}{ c | c }
  Texture           & heavily arpeggiated \\ 
  Tonality          & Minimal Modulation \\
  Text Illustration & N/A\\
\end{tabular}



\section*{Rhythm}

\begin{tabular}{ c | c }
  Rhythmic Pattern & steady 8th notes with decoration\\
                   & mismatched duplets and triplets\\
  Tempo            & Schnell (Fast)\\
\end{tabular}



\section*{The Piano Component}

\begin{tabular}{ c | c }
  Preludes/Interludes/Postludes & simple I - V introduction\\ 
  Tonality       & conventional CPP harmony\\
  Use of Motives & N/A\\
\end{tabular}

\clearpage

\section*{Poem/Text}

\begin{large}
  \textbf{\textit{Der Contrabandiste} - Emanuel von Geibel }
\end{large}

\vspace{10pt}

\begin{Parallel}[v]{0.48\textwidth}{0.48\textwidth}
  \ParallelLText{
    Ich bin der Contrabandiste, \\
    Weiß wohl Respekt mir zu schaffen. \\
    Allen zu trotzen, ich weiß es, \\
    Furcht nur, die hab' ich vor keinem. \\
    Drum nur lustig, nur lustig! \\
 \\
    Wer kauft Seide, Tabak! \\
    Ja wahrlich, mein Rößlein ist müde, \\
    Ich eil', ja eile, \\
    Sonst faßt mich noch gar die Runde, \\
    Los geht der Spektakel dann. \\
    Lauf nur zu, mein lustiges Pferdchen, \\
    Ach, mein liebes, gutes Pferdchen, \\
    Weißt ja davon, mich zu tragen! \\
  }\

  \ParallelRText{
    I am the smuggler, \\
    And know well how to inspire respect; \\
    I know how to defy everyone, \\
    and I fear no one. \\
    So let us be merry! \\
    Who shall buy my silk and tobacco? \\
    Tryly, my little horse is tired, \\
    I hurry, yes, hurry, \\
    Otherwise the patrol will catch me, \\
    And then things will go very badly! \\
    Run, my merry horse, \\
    Ah, my dear good steed, \\
    You know well how to carry me!
  }\
\end{Parallel}


\section*{Poet}


I couldn't find much information of the Poet, simply that 
he is known for this poem alone.

\section*{Choice of Text} 
Schubert has multiple songs and other works arranging this
text. Each of his settings occupies a unique soundworld and 
emotional context. This setting, compared to his more/ dramatic 
and popular - at least amongst pianists - D.493, is more calm 
and reticent, lending to a figure of a weary, wise, and well 
travelled \textit{Wanderer}.


\section*{Prosody}
I'm not familiar with much German diction, so it seems very 
normal/characteristically German in pronunciation and stress 
to me.

% \nocite{*}
% \printbibliography
\end{document}
