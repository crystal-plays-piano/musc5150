\documentclass{article}

% - imports
\usepackage{enumitem}
\usepackage[utf8]{inputenc}
\usepackage[backend=biber]{biblatex-chicago}
\addbibresource{bibliography.bib}
\usepackage{url}
\usepackage{parskip}
\usepackage{indentfirst}
\usepackage{parallel}

\title{\vspace*{-72pt} Title}
\author{Crystal Mandal}
\date{Last Edited \today}

\begin{document}

  \maketitle

  \section*{Question 1}
    Poet Chosen: Francesco Petrarca \\
    Francesco Petrarca (1304 - 1374), commonly known by the mononym
    Petrarch, numbers among the most influential poets in the Italian
    Language; ironic, as his Humanist and Catholic
    background informs a relative
    disdain for the ``vulgar" language. \autocite{petrarch:brittanica}
    At the age of 23, Petrarch
    meets ``Laura", the love of his life, whom he immortalises as the
    subject in the over
    300 sonnets (and other poems) that comprise his collection of
    Italian Poetry ``Canzoniere" - meaning ``songbook". The Critic
    Robert Stanley Martin asserts that Petrarch
    ``reimagines the conventions of love poetry in the most profound way"
    by equating romance and Godly devotion. \autocite{poetryfoundation}

  \section*{Question 2}
    \subsection*{Part C}
      Both Settings by Verdi are clearly drawing from the same
      literary inspiration as Schubert. \textit{Deh, pietoso}'s
      Schubertian Counterpart remains unfinished, so it is
      more difficult to compare the two; luckily,
      \textit{Perduta ho la pace}
      can be placed adajacent and equal to \textit{Gretchen am Spinnrade}.
      It's quite amusing to see both Verdi and Schubert settings have
      ``spinning" textures in the piano part, though the actual textures
      themselves are quite different. The Verdi setting has a more
      irregular ostinato in triple meter that starts as eigth notes and
      slowly increases in speed to 16th and then to tremolo figures.
      Where the Schubert setting has the memorable single fermata on
      \textit{"sein Kuss!"}, Verdi uses fermata throughout the piece
      to imply a more regular starting and stopping gesture.

  \section*{Question 3}
    I think most of/many of the songs on this list would fit her
    definition. I'm not sure I could quite articulate what makes a
    melody ``Italianate", but think I could definitely pick out an Italian
    melody from a German or (in some cases) an American melody.
    None of the pieces on the list seem especially non-Italian to me
    (though I won't deny that the Italian language song informs the
    ``Italianate" melody), but I do believe some of the slower and more
    dramatic songs - like Donizetti's \textit{Amore marinato} and
    Verdi's \textit{Deh, pietoso} - seem less compliant to Kimball's
    definition. Then again, the slow movents of Bach's keyboard
    concerti and some of his more ornate Sarabandes are no less
    ``Italian" in style for the keyboard than the quicker, bouncier
    pieces (like the correntes from the 5th French Suite or the
    6th Partita).

  \section*{Question 4}
    The Respighi setting is so immediately, if I were to invent a word
    for a second, \textit{Respighi-esque}. The rippling arpeggios
    and lyrical melody above (also, quite curiously similar to
    Debussy's famous \textit{Suite bergamasque} composed just a
    few years earlier) and the careful elusion of the underlying dance
    until the very end all mirror structures in Respighi's own piano
    suites and arias - or, at least, all of the pieces that I have
    encountered. I quite prefer the more dramatic treatment of the text
    by Respighi compared to the more ``bouncy", dance-like nature
    of the Tosti setting.

  \section*{Question 5}
    Both of these songs have sung parts that linger around very stable
    chords and chord tones - in contrast to the very difficult Debussy
    pieces we have looked at - and very straightforward harmonic
    progressions. I, as an untrained vocalist, was able to sight-sing
    both immediately. The extreme parts of the vocalist's range are also
    treated with a more prominent part in the texture and given a louder
    dynamic, making them, as far as I understand, easier to sing.

  \section*{Question 6}
    A stornello is a common short Italian poetic form that
    uses an AAA rhyme scheme for each 3-line stanza. Common
    further restrictions include the usage of 10- or 11-syllable
    lines and the usage of a slant rhyme in the second line (that is,
    an AaA rhyme scheme).\autocite{sornello:wordwool}
    I'm not sure how accurate this definition is.
    While it seems like a well researched definition, and it seems to
    work in practice, I'm not sure how much I can trust the source I
    was able to find this comprehensive definition of the stornello.
    \begin{enumerate}[label=\alph*)]
      \item{Verdi's Stornello follows the ``triplet" rhythm rule (the 10-
        and 11- syllable lines) but not the rhyme scheme or the length.}
      \item{Again, Tosti's stornello has the same triplet rhythm, but
        doesn't follow the rules of rhyme scheme.}
      \item{Cimara's stornello again has the triplet rhythm but not
        the rhyme scheme. }

    \end{enumerate}
    It is important to note that, while the three songs do not fit the
    strictest rules that I found for the \textit{stornello}, they do
    all use the same lilting, triplet rhythm in their text, as well as
    an adherence to simple ABAB rhyme schemes and some amusing slant
    rhymes.

  \section*{Question 7}
    I'm not yet sure, other than the requirement of a pre-composed
    poem/text, what, musically, makes an art song distinct. These
    tracks seem, however, more immediately complex

    % 
\vspace*{-72pt}

\begin{Large}
  Song Sheet: F. Schubert's \textit{Der Wanderer} 
\end{Large}


% \begin{enumerate}[label=\Alph*.)]
%   \item{The Chosen Song is: Franz Schubert's \textit{Der Wanderer}, Op. 65, No., D.649 }
% \end{enumerate}

\begin{itemize}[label=]
  \item{\textbf{Title: \textit{Der Wanderer}, Op. 65, No. 2, D. 649}}
  \item{\textbf{Composed: February 1819}}
  \item{\textbf{Composer: Franz Schubert (1797-1828)}}
  \item{\textbf{Poet: Georg Philipp Schmidt ('von Lübeck', 1766-1849) formerly attributed to Zacharias Werner }}\autocite{imslp-649}
\end{itemize}



\section*{General}
This song is one of four composed by Schubert on the same poem. 
The other three are: D.649, D.795, and D.8D.649, D.795, and D.870 \autocite{tapatalk} ; 
of note, D.649 is the most well known (at least, amongst pianists) of 
the bunch, and is - quite famously - the primary subject of the great 
\textit{Fantasy in C Major} (or ``Wanderer Fantasy") of Schubert.

Approximate Performance Time: a little under 4 minutes.

\section*{Melody}

\begin{tabular}{ c | c }
  Melodic Contour    & mostly stepwise or by thirds \\ 
  Tessitura          & spans a Major 9th, from C4 to D5\\
  Vocal Articulation & I'm not sure what this means\\
  Text Illustration  & N/A
\end{tabular}



\section*{Harmony}

\begin{tabular}{ c | c }
  Texture           & Chorale Texture \\ 
  Tonality          & Minimal Modulation \\
                    & Some Chromatic Passages\\
  Text Illustration & N/A\\
\end{tabular}



\section*{Rhythm}

\begin{tabular}{ c | c }
  Rhythmic Pattern & Flowing Chorale in 8th notes \\
                   & few ``marching" interjections\\ 
  Tempo & Andante\\
\end{tabular}



\section*{The Piano Component}

\begin{tabular}{ c | c }
  Preludes/Interludes/Postludes & few solo piano passages\\ 
  Tonality       & Some chromatic passages \\
                 & heavy insistence on a melodic G\#\\
  Use of Motives & there are many references to \\
                 & the opening piano gesture\\
\end{tabular}



\section*{Poem/Text}

\begin{large}
  \textbf{\textit{Der Wanderer} - Georg Philipp Schmidt 'von Lübeck' }
\end{large}

\vspace{10pt}

\begin{Parallel}[v]{0.48\textwidth}{0.48\textwidth}
  \ParallelLText{
    Wie deutlich des Mondes Licht \\
    Zu mir spricht, \\
    Mich beseelend zu der Reise:\\
    "Folge treu dem alten Gleise,\\
    Wähle keine Heimath nicht.\\
    Ew'ge Plage\\
    Bringen sonst die schweren Tage.\\
    Fort zu andern\\
    Sollst du wechseln, sollst du wandern,\\
    Leicht entfliehend jeder Klage."\\

    Sanfte Ebb' und hohe Fluth,\\
    Tief im Muth,\\
    Wandr' ich so im [Dunkel]1 weiter,\\
    Steige muthig, singe heiter,\\
    Und die Welt erscheint mir gut.\\
    Alles reine\\
    Seh' ich mild im Wiederscheine,\\
    Nichts verworren\\
    In des Tages Gluth verdorren:\\
    Froh umgeben, doch alleine.}\

  \ParallelRText{
    How clearly the moon's light \\
    Speaks to me, \\
    Inspiring me to journey; \\
    "Follow truly the ancient path, \\
    Choose no homeland whatsoever. \\
    Otherwise the heavy days bring \\
    Endless troubles ; \\
    Away, to the other \\
    Should you change, should you wander, \\
    Lightly shedding every woe." \\

    Gentle ebb and lofty flood, \\
    Deep in courage, \\
    I wander farther in darkness, \\
    I climb bravely, singing cheerfully, \\
    And the world seems good to me. \\
    All pureness \\
    See I softly in the twilight, \\
    Without confusion \\
    Fading in the day's afterglow: \\
    Surrounded by joy, but alone.} 
\end{Parallel}


\section*{Poet}


I couldn't find much information of the Poet, simply that 
he is known for this poem alone.

\section*{Choice of Text} 
Schubert has multiple songs and other works arranging this
text. Each of his settings occupies a unique soundworld and 
emotional context. This setting, compared to his more/ dramatic 
and popular - at least amongst pianists - D.493, is more calm 
and reticent, lending to a figure of a weary, wise, and well 
travelled \textit{Wanderer}.


\section*{Prosody}
I'm not familiar with much German diction, so it seems very 
normal/characteristically German in pronunciation and stress 
to me.

  % 
\vspace*{-72pt}

\begin{Large}
  Song Sheet: R. Schumann's \textit{Der Contrabandista} 
\end{Large}


% \begin{enumerate}[label=\Alph*.)]
%   \item{The Chosen Song is: Franz Schubert's \textit{Der Wanderer}, Op. 65, No., D.649 }
% \end{enumerate}

\begin{itemize}[label=]
  \item{\textbf{Title: \textit{Der Contrabandiste}, Op. 74 }} 
  \item{\textbf{Composer: Robert Schumann (1810-1856)}}
  \item{\textbf{Poet: Emanuel von Geibel (1815 - 1884) }}\autocite{imslp-74}
\end{itemize}



\section*{General}

This song is from the appendix to R. Schumann's 
\textit{Spanisches Liederspeiel}. According to 
Hyperion records, it was removed from the original 
set due to its relative unimportance to the storyline 
of the overall set.



Approximate Performance Time: a little over 1 minute

\section*{Melody}

\begin{tabular}{ c | c }
  Melodic Contour    & very chordal, lots of jumps\\ 
  Tessitura          & almost 2 octaves, from A2 to G4\\
  Vocal Articulation & I'm not sure what this means\\
  Text Illustration  & N/A
\end{tabular}



\section*{Harmony}

\begin{tabular}{ c | c }
  Texture           & heavily arpeggiated \\ 
  Tonality          & Minimal Modulation \\
  Text Illustration & N/A\\
\end{tabular}



\section*{Rhythm}

\begin{tabular}{ c | c }
  Rhythmic Pattern & steady 8th notes with decoration\\
                   & mismatched duplets and triplets\\
  Tempo            & Schnell (Fast)\\
\end{tabular}



\section*{The Piano Component}

\begin{tabular}{ c | c }
  Preludes/Interludes/Postludes & simple I - V introduction\\ 
  Tonality       & conventional CPP harmony\\
  Use of Motives & N/A\\
\end{tabular}

\clearpage

\section*{Poem/Text}

\begin{large}
  \textbf{\textit{Der Contrabandiste} - Emanuel von Geibel }
\end{large}

\vspace{10pt}

\begin{Parallel}[v]{0.48\textwidth}{0.48\textwidth}
  \ParallelLText{
    Ich bin der Contrabandiste, \\
    Weiß wohl Respekt mir zu schaffen. \\
    Allen zu trotzen, ich weiß es, \\
    Furcht nur, die hab' ich vor keinem. \\
    Drum nur lustig, nur lustig! \\
 \\
    Wer kauft Seide, Tabak! \\
    Ja wahrlich, mein Rößlein ist müde, \\
    Ich eil', ja eile, \\
    Sonst faßt mich noch gar die Runde, \\
    Los geht der Spektakel dann. \\
    Lauf nur zu, mein lustiges Pferdchen, \\
    Ach, mein liebes, gutes Pferdchen, \\
    Weißt ja davon, mich zu tragen! \\
  }\

  \ParallelRText{
    I am the smuggler, \\
    And know well how to inspire respect; \\
    I know how to defy everyone, \\
    and I fear no one. \\
    So let us be merry! \\
    Who shall buy my silk and tobacco? \\
    Tryly, my little horse is tired, \\
    I hurry, yes, hurry, \\
    Otherwise the patrol will catch me, \\
    And then things will go very badly! \\
    Run, my merry horse, \\
    Ah, my dear good steed, \\
    You know well how to carry me!
  }\
\end{Parallel}


\section*{Poet}


I couldn't find much information of the Poet, simply that 
he is known for this poem alone.

\section*{Choice of Text} 
Schubert has multiple songs and other works arranging this
text. Each of his settings occupies a unique soundworld and 
emotional context. This setting, compared to his more/ dramatic 
and popular - at least amongst pianists - D.493, is more calm 
and reticent, lending to a figure of a weary, wise, and well 
travelled \textit{Wanderer}.


\section*{Prosody}
I'm not familiar with much German diction, so it seems very 
normal/characteristically German in pronunciation and stress 
to me.

% \nocite{*}
% \printbibliography
\end{document}
