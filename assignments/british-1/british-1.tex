\documentclass[12pt,letterpaper]{article}

% - default packages

\usepackage[doublespacing]{setspace}
\usepackage{indentfirst}
\usepackage{parskip}
\usepackage{enumitem}

\title{\vspace*{-72pt}British 1}
\author{Crystal Mandal}
\date{Last Edited \today}

\begin{document}

\maketitle

\section*{Question 1}
I notice that all three songs start with a short but substantial introduction with
a clear sequence of dominant and tonicized chords, making the tonality and starting
pitches for the vocalist very clear. The verses are also separated by very strict
introduction patterns and repeating melodic gestures, but commonly also have a
small textural change. The piano parts are also quite supportive: all three songs
give the piano minimal melodic material, and the harmonic textures are all simple
chorale parts or simple arpeggios.

Side note: I did not know Samuel Coleridge-Taylor was British.

\clearpage
\section*{Question 2}

\begin{enumerate}[label=(\alph*)]
  \item{Compared to the Mahler and Brahms orchestrations we heard, these seem similarly grand
    but with less of a vocalist-first composition. I found both pieces beautiful but quite difficult
  to pick out the vocalist. I would assume the intent is for the full ensemble colour more than
  the effect of an orchestra accompanying a vocalist. }
  \item{I think the information of the painting is fun, but, unless the recital or recital group
  had more songs with affiliated paintings, I wouldn't include the information in a programme note. }
\end{enumerate}


\clearpage
\section*{Question 3}
My first thought is that Britten's ``Salley Gardens" sounds strikingly like the Scottish
folk song ``Loch Lomond", which is quite dear to me. The song ``Pray Goody" is delightful!
I can't believe I hadn't heard it before. I can't say I'm especially fond of ``When You're
Feeling Like Expressing Your Affection", but it's cute and short and would make for a great
pre-recital or post-encore piece in a more casual setting. The Sonnet and the ``Nurse's Song"
are more in line with the style of music I immediately associate with Britten. I'm always
intrigued, and Britten is a brilliant composer that I wish enticed me more.


\clearpage
\section*{Question 4}
Yes, firmly. I think this is probably my favourite song we've heard this semester.
There's so much artistry in the voice and piano and the organisation of the two
I find it breathtaking. My justification of this piece as art song comes in relation
to other pieces I justify as art song and put in conversation with this one. The
most interesting of the pieces we've looked at this semester that I think can
be put in discourse with ``Life Story" is William Bolcom's ``Amor". If that song
(and the containing set) is firmly in the world of Art Song - though it sidesteps
plenty of the expectations of the category - then I cannot find reason why, in a
similar vein, this cannot be Art Song.



\end{document}
