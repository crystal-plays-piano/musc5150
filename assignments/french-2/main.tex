\documentclass{article}

% - imports
\usepackage{enumitem}
% \usepackage[backend=biber]{biblatex-chicago}
% \addbibresource{bibliography.bib}
\usepackage{url}
\usepackage{parskip}
\usepackage{indentfirst}
\usepackage[utf8]{inputenc}
\usepackage{parallel}

\title{\vspace*{-72pt} French 2 }
\author{Crystal Mandal}
\date{Last Edited \today}

\begin{document}

  \maketitle

  \section*{Question 1}
  Assigned Poet: Charles Pierre Baudelaire
  \begin{itemize}
    \item{Dates: 9 April 1821 - 31 August 1867}
    \item{Bio}
      \begin{itemize}
        \item{Father passed at age 10, mother remarried in a year}
        \item{``aimless" in life: travelled to Kolkata in 1839}
        \item{Published \textit{Les Fleurs du mal}, first collection of
          poetry, in 1857 (age 36)}
        \item{suffered a stroke in 1866, passed in 1867 - stroke was
          likely caused by a lifetime use/abuse of laudanum(opium) and
          alcohol.}
      \end{itemize}
    \item{Best Known For}
      \begin{itemize}
        \item{\textit{Les Fleurs du mal}, 1857 (The Flowers of Evil)}
        \item{\textit{Petits poèmes en prose}, 1868 (Little Prose Poems)}
      \end{itemize}
    \item{Works Set by Debussy (Acc. to ``A French Song Companion"}
      \begin{itemize}
        \item{\textit{Le balcon} (The balcony)}
        \item{\textit{Harmonie du soir} (Evening harmony)}
        \item{\textit{Le jet d'eau} (The fountain)}
        \item{\textit{Recuillement} (Meditation)}
        \item{\textit{La mord des amants} (The death of lovers)}
        \item{Lieder [dot] net has a few more entries, totalling
          ten works set by Debussy}
      \end{itemize}
    \item{Approximately 135 texts in total have been set by composers
      in approximately 438 settings (according to Lieder [dot] net)}
    \item{Harmonie du soir}
      \begin{itemize}
        \item{almost obsessively frugal in motivic material}
        \item{striking, inventive harmonies}
      \end{itemize}
    \clearpage{}
    \item{Le jet d'eau}
      \begin{itemize}
        \item{reminds me of Ravel's \textit{Jeux d'eau} }
        \item{rhythmically playful}
        \item{countermelody in piano crossing hands with quicker,
          textural figuration - reminiscent of Debussy's own
          \textit{Images}}
      \end{itemize}
  \end{itemize}

  \clearpage{}

  \section*{Question 2}
    For this Question, I could not find an orchestrated version of
    Hahn's setting, so I compared the Fauré setting for which I could
    find the version with the string quartet.

    Notably, I could not find a score for the string parts of this
    piece, and it seems, as detailled in the liner notes of the
    Columbia Masterworks \textit{Chamber Music from Marlboro} CD,
    that, while the performance with string quartet seems intentional
    by Fauré, the printed score has been lost.
    \subsection*{Piano Accompaniment}
      \begin{itemize}
        \item{a very bright arrangement + sound}
        \item{the triplet piano figure could be a reference to Beethoven's
          \textit{Moonlight} Sonata}
        \item{very pianistic}
      \end{itemize}
    \subsection*{With String Quartet}
      \begin{itemize}
        \item{due to the instrumentation, \textit{much} warmer}
        \item{countermelodies taken by different instruments}
        \item{very indicative of Fauré's ``stillness" in harmonic contexts}
      \end{itemize}

    Overall, though the piano has my heart, I do prefer to hear the
    version with
    both piano and string quartet. The added warmth, colour, and clarity
    of countermelody are most appreciated. I will admit, however, that
    the quiet intimacy of a simple piano accompaniment is lost in a
    larger ensemble; this is - like the contrast between R. Schumann's
    \textit{Widmung} and the Liszt arrangement - the most important
    deciding factor outside of sound. Is it ``worth it" to lose out on
    the intimacy factor for a warmer sound? I think it is, but only by
    a small margin.

  \clearpage{}

  \section*{Question 3}
    \subsection*{Debussy - 1882 version}
      \begin{itemize}
        \item{dramatic, passionate}
        \item{conventional chordal piano texture with lilting rhythm}
        \item{less `muted' than other settings}
      \end{itemize}
    \subsection*{Debussy - Fêtes galantes}
      \begin{itemize}
        \item{pentatonic-adjacent melody}
        \item{interesting "rocking" piano texture at
          "Au souffle berceur t doux"}
        \item{never louder than piano}
      \end{itemize}
    \subsection*{Fauré}
      \begin{itemize}
        \item{piano counter melody! }
        \item{arpeggiated figuration reminiscent of Mendelssohn}
        \item{LOTS of suspensions used for colour (left unresolved)}
      \end{itemize}
    \subsection*{Hahn}
      \begin{itemize}
        \item{hilariously, the piano motive reminds me of Gershwin's
          `Summertime' from Porgy and Bess, which, interestingly
          enough, has a similarly slow/somber tempo and sound (though
          the lyrics wildly differ in tone).}
        \item{beautiful piano counter melod\textbf{ies}.}
        \item{harmonic stability/stillness (the harmony rarely changes)}
      \end{itemize}
    \subsection*{Poldowski}
      \begin{itemize}
        \item{HUGE dynamic range}
        \item{each stanza is given a distinct treatment (new textures
          and melodies)}
        \item{reminds me of the first movement of Beethoven Op 101
          or the second movement of Bach's 5th Brandenburg Concerto
          (both movements could, by Shenker, be reduced to a simple,
          though heavily extended, V-I Cadence)}
      \end{itemize}

  \clearpage{}

    \subsection*{Similarities}
      \begin{itemize}
        \item{very slow (tempo and rhythmic divisions)}
        \item{slow harmonic movement + unresolved `non-tonal' figures}
        \item{each composer likes to keep the melody on one note
          for a long time (be it repetition or simply meandering
          around a particular note.}
      \end{itemize}

  % 
\vspace*{-72pt}

\begin{Large}
  Song Sheet: F. Schubert's \textit{Der Wanderer} 
\end{Large}


% \begin{enumerate}[label=\Alph*.)]
%   \item{The Chosen Song is: Franz Schubert's \textit{Der Wanderer}, Op. 65, No., D.649 }
% \end{enumerate}

\begin{itemize}[label=]
  \item{\textbf{Title: \textit{Der Wanderer}, Op. 65, No. 2, D. 649}}
  \item{\textbf{Composed: February 1819}}
  \item{\textbf{Composer: Franz Schubert (1797-1828)}}
  \item{\textbf{Poet: Georg Philipp Schmidt ('von Lübeck', 1766-1849) formerly attributed to Zacharias Werner }}\autocite{imslp-649}
\end{itemize}



\section*{General}
This song is one of four composed by Schubert on the same poem. 
The other three are: D.649, D.795, and D.8D.649, D.795, and D.870 \autocite{tapatalk} ; 
of note, D.649 is the most well known (at least, amongst pianists) of 
the bunch, and is - quite famously - the primary subject of the great 
\textit{Fantasy in C Major} (or ``Wanderer Fantasy") of Schubert.

Approximate Performance Time: a little under 4 minutes.

\section*{Melody}

\begin{tabular}{ c | c }
  Melodic Contour    & mostly stepwise or by thirds \\ 
  Tessitura          & spans a Major 9th, from C4 to D5\\
  Vocal Articulation & I'm not sure what this means\\
  Text Illustration  & N/A
\end{tabular}



\section*{Harmony}

\begin{tabular}{ c | c }
  Texture           & Chorale Texture \\ 
  Tonality          & Minimal Modulation \\
                    & Some Chromatic Passages\\
  Text Illustration & N/A\\
\end{tabular}



\section*{Rhythm}

\begin{tabular}{ c | c }
  Rhythmic Pattern & Flowing Chorale in 8th notes \\
                   & few ``marching" interjections\\ 
  Tempo & Andante\\
\end{tabular}



\section*{The Piano Component}

\begin{tabular}{ c | c }
  Preludes/Interludes/Postludes & few solo piano passages\\ 
  Tonality       & Some chromatic passages \\
                 & heavy insistence on a melodic G\#\\
  Use of Motives & there are many references to \\
                 & the opening piano gesture\\
\end{tabular}



\section*{Poem/Text}

\begin{large}
  \textbf{\textit{Der Wanderer} - Georg Philipp Schmidt 'von Lübeck' }
\end{large}

\vspace{10pt}

\begin{Parallel}[v]{0.48\textwidth}{0.48\textwidth}
  \ParallelLText{
    Wie deutlich des Mondes Licht \\
    Zu mir spricht, \\
    Mich beseelend zu der Reise:\\
    "Folge treu dem alten Gleise,\\
    Wähle keine Heimath nicht.\\
    Ew'ge Plage\\
    Bringen sonst die schweren Tage.\\
    Fort zu andern\\
    Sollst du wechseln, sollst du wandern,\\
    Leicht entfliehend jeder Klage."\\

    Sanfte Ebb' und hohe Fluth,\\
    Tief im Muth,\\
    Wandr' ich so im [Dunkel]1 weiter,\\
    Steige muthig, singe heiter,\\
    Und die Welt erscheint mir gut.\\
    Alles reine\\
    Seh' ich mild im Wiederscheine,\\
    Nichts verworren\\
    In des Tages Gluth verdorren:\\
    Froh umgeben, doch alleine.}\

  \ParallelRText{
    How clearly the moon's light \\
    Speaks to me, \\
    Inspiring me to journey; \\
    "Follow truly the ancient path, \\
    Choose no homeland whatsoever. \\
    Otherwise the heavy days bring \\
    Endless troubles ; \\
    Away, to the other \\
    Should you change, should you wander, \\
    Lightly shedding every woe." \\

    Gentle ebb and lofty flood, \\
    Deep in courage, \\
    I wander farther in darkness, \\
    I climb bravely, singing cheerfully, \\
    And the world seems good to me. \\
    All pureness \\
    See I softly in the twilight, \\
    Without confusion \\
    Fading in the day's afterglow: \\
    Surrounded by joy, but alone.} 
\end{Parallel}


\section*{Poet}


I couldn't find much information of the Poet, simply that 
he is known for this poem alone.

\section*{Choice of Text} 
Schubert has multiple songs and other works arranging this
text. Each of his settings occupies a unique soundworld and 
emotional context. This setting, compared to his more/ dramatic 
and popular - at least amongst pianists - D.493, is more calm 
and reticent, lending to a figure of a weary, wise, and well 
travelled \textit{Wanderer}.


\section*{Prosody}
I'm not familiar with much German diction, so it seems very 
normal/characteristically German in pronunciation and stress 
to me.

  % 
\vspace*{-72pt}

\begin{Large}
  Song Sheet: R. Schumann's \textit{Der Contrabandista} 
\end{Large}


% \begin{enumerate}[label=\Alph*.)]
%   \item{The Chosen Song is: Franz Schubert's \textit{Der Wanderer}, Op. 65, No., D.649 }
% \end{enumerate}

\begin{itemize}[label=]
  \item{\textbf{Title: \textit{Der Contrabandiste}, Op. 74 }} 
  \item{\textbf{Composer: Robert Schumann (1810-1856)}}
  \item{\textbf{Poet: Emanuel von Geibel (1815 - 1884) }}\autocite{imslp-74}
\end{itemize}



\section*{General}

This song is from the appendix to R. Schumann's 
\textit{Spanisches Liederspeiel}. According to 
Hyperion records, it was removed from the original 
set due to its relative unimportance to the storyline 
of the overall set.



Approximate Performance Time: a little over 1 minute

\section*{Melody}

\begin{tabular}{ c | c }
  Melodic Contour    & very chordal, lots of jumps\\ 
  Tessitura          & almost 2 octaves, from A2 to G4\\
  Vocal Articulation & I'm not sure what this means\\
  Text Illustration  & N/A
\end{tabular}



\section*{Harmony}

\begin{tabular}{ c | c }
  Texture           & heavily arpeggiated \\ 
  Tonality          & Minimal Modulation \\
  Text Illustration & N/A\\
\end{tabular}



\section*{Rhythm}

\begin{tabular}{ c | c }
  Rhythmic Pattern & steady 8th notes with decoration\\
                   & mismatched duplets and triplets\\
  Tempo            & Schnell (Fast)\\
\end{tabular}



\section*{The Piano Component}

\begin{tabular}{ c | c }
  Preludes/Interludes/Postludes & simple I - V introduction\\ 
  Tonality       & conventional CPP harmony\\
  Use of Motives & N/A\\
\end{tabular}

\clearpage

\section*{Poem/Text}

\begin{large}
  \textbf{\textit{Der Contrabandiste} - Emanuel von Geibel }
\end{large}

\vspace{10pt}

\begin{Parallel}[v]{0.48\textwidth}{0.48\textwidth}
  \ParallelLText{
    Ich bin der Contrabandiste, \\
    Weiß wohl Respekt mir zu schaffen. \\
    Allen zu trotzen, ich weiß es, \\
    Furcht nur, die hab' ich vor keinem. \\
    Drum nur lustig, nur lustig! \\
 \\
    Wer kauft Seide, Tabak! \\
    Ja wahrlich, mein Rößlein ist müde, \\
    Ich eil', ja eile, \\
    Sonst faßt mich noch gar die Runde, \\
    Los geht der Spektakel dann. \\
    Lauf nur zu, mein lustiges Pferdchen, \\
    Ach, mein liebes, gutes Pferdchen, \\
    Weißt ja davon, mich zu tragen! \\
  }\

  \ParallelRText{
    I am the smuggler, \\
    And know well how to inspire respect; \\
    I know how to defy everyone, \\
    and I fear no one. \\
    So let us be merry! \\
    Who shall buy my silk and tobacco? \\
    Tryly, my little horse is tired, \\
    I hurry, yes, hurry, \\
    Otherwise the patrol will catch me, \\
    And then things will go very badly! \\
    Run, my merry horse, \\
    Ah, my dear good steed, \\
    You know well how to carry me!
  }\
\end{Parallel}


\section*{Poet}


I couldn't find much information of the Poet, simply that 
he is known for this poem alone.

\section*{Choice of Text} 
Schubert has multiple songs and other works arranging this
text. Each of his settings occupies a unique soundworld and 
emotional context. This setting, compared to his more/ dramatic 
and popular - at least amongst pianists - D.493, is more calm 
and reticent, lending to a figure of a weary, wise, and well 
travelled \textit{Wanderer}.


\section*{Prosody}
I'm not familiar with much German diction, so it seems very 
normal/characteristically German in pronunciation and stress 
to me.

% \nocite{*}
% \printbibliography
\end{document}
