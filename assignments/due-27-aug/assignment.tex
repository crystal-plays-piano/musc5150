\documentclass{article}
% - imports
\usepackage{enumitem}

\title{\vspace*{-72pt} Assignment 2}
\author{Crystal Mandal}
\date{Last Edited \today}

\begin{document}

\maketitle

\begin{enumerate}
  \item \begin{enumerate}[label=(\alph*)]
    \item I feel as though I am quite familiar with the intersectionality of music 
    (specifically song) and poetry. Much of my final undergraduate year of study 
    was focused on exploring this sphere of thought.
    
    \item While I am familiar with a wide variety of ergodic literature - primarily 
    in the form of literary novels and essays - I am much less familiar with ergodic 
    poetry, so the ``calligramme" form was unfamiliar (and exciting!) to me. 
    
    \item I am unfortunately not working on any lyric or poetic music, nor am I at 
    this time continuing my exploration of incarcerated writing, so these concepts 
    are a little disconnected from my work and studies.
    
    \item I do not currently have any questions.

  \end{enumerate}


\item \begin{enumerate}
\item The poem to which Haydn's ``Pastoral Song" is set, ``My Mother Bids Me Bind My Hair" is evocative to me in a way I cannot quite describe.

    \item I'm most interested in the lyric rhythm, and I'm always easily moved by pastoral scenes and music.

    \item As with other music imitating the spinning of thread, gentle arpeggios in 6/8 accompany a flowing and ``simple" melody.
\end{enumerate} 

\item The voice laments ``while Lubin is away" multiple times (which isn't in the poem's text). This is curious to me because I could \textit{not} find any writing abut \textit{who} Lubin is.

\item I'm a big fan of the musical jokes in ``A Mother's Tardy Arrival."
 
\item Unfortunately, I am not a vocalist, and of the languages I speak, only English is represented in these selections, so "My Mother Bid Me Bind My Hair" would be easiest to perform.

\item I would assign "A Mother's Tardy Arrival". Much of the collection is focused on love and romance, or questions about God, Life, and Death. A child will likely not have those experiences, but will definitely know how it feels to be waiting on a Guardian to pick them up. 

\item Damoetas, according to Oxford Reference library, likely refers to a shepherd from Virgil's writing. I'm not especially familiar with Virgil, so I cannot corroborate this claim. Amusingly, Damoetas is also a type of Australian jumping spider; this is almost certainly not the correct interpretation, but I am quite amused by the image of a child being thoroughly transfixed by the sight of a spider, bitten, and then needing their mother to assist them.

\item I am not a lyrics person. I can not and will not be able to discern what words are being sung and gain any meaning from that. So, hearing the song sung in two languages I do not understand doesn't change too much for me, though I do prefer the quicker pace of the higher-pitch vocalist. Of the translations, however, I prefer the German: the explicit reference to the \textit{forget me not} flower is sweet.


\item The first set of program notes is, to put it kindly, an unstructured and unhelpful mess. While context can be helpful, the notes do nothing to simplify the reception of the music. These notes, to me, are an active detriment to the enjoyment of the music. In contrast, the notes for Op. 94 are lovely! By outlining the musically interesting moments and commenting on the structure, the notes act as a roadmap for the listener, facilitating a greater understanding of the music.


\item As recounted in his letters, it can be inferred that Beethoven composed this piece in response to the rejection of a lover. The dramatic nature of this text is, to me, an exaggerated response.


\item As a pianist, I will unfortunately lament the lack of especially interesting accompaniment parts, though I am definitely interested in finding a vocalist to perform Beethoven's Op. 94. The operatic, narrative style and the significant contrast between episodes looks like a delight to learn and put together.



\end{enumerate}
\end{document}
