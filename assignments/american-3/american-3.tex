\documentclass[12pt]{article}

% - default packages

\usepackage[backend=biber]{biblatex-chicago}
\usepackage[doublespacing]{setspace}
\usepackage{indentfirst}
\usepackage{parskip}

\addbibresource{bibliography.bib}

\title{\vspace*{-72pt} American III}
\author{Crystal Mandal}
\date{Last Edited \today}

\begin{document}

  \maketitle

  \section*{Question 1}
  José Garcia Villa's style, as described by both the JSTOR Daily Article \autocite{jstor-villa-overview}
  and by the Encyclopedia of World Biography\autocite{encyclopedia-jose-garcia}, is delicate, yet forceful,
  and innovative. He invented the comma poem, and was hugely influential in the development of the form.
  Among his other innovations is the ``reversed consonance" rhyme scheme, which has a distinct sound and
  rhythm as compared to other, more conventional rhyme patterns. I find that both of the poems ``Moonlight's
  Watermelon" and ``Have Come, Am Here", are striking in their directness and clarity. The comma usage throughout
  ``Moonlight's Watermelon" is compliant with the style guidelined by the encyclopedia articles, but the
  ``reversed consonance" rhyming scheme is not used in either poem - or, if it is, it is \textit{very} well
  hidden.

  \clearpage
  \section*{Question 2}
  What strikes me about Barber's earlier work is that, while still portraying some of the very open intimacy
  of his later works, the harmonic language is so wildly different it's a little jarring. Both ``The Daisies"
  and the Op. 13 songs (written both before his Excursions Op. 20, the earliest piano work I have studied)
  are relatively straightforward, tonal, and functional in harmony. Amusing, too, is the rhythmic playfulness;
  Barber's work, to me, is characterised by a shifting meter and lots of dramatic pauses and moments of
  rhythmic uncertainty. This sort of playfulness, while still present, is much less apparent in the earlier
  songs. Curiously, I think his melodic content remains - shall we say? - \textit{Barber-esque}. The focus
  on pentatonic and diatonic movement, simple intervals, and consonant leaps is present in both the earlier
  songs and in the Op. 45 songs (that I listened to), as well as reflected in the respective Op.20 Excursions
  and Op.46 Ballade.

  \clearpage
  \section*{Question 3}
  This recital group might seem odd. I did not immediately see the Hundley song as one perfect for marriage,
  but instead one perfect for a funeral - that is, sung for a love that is lost. In that vein, I have chosen
  three songs for their sound before their text - and in one case an extramusical context.
  \begin{enumerate}
    \item{ Gary Bachlund* - Warm Summer Sun}
    \item{ Laitman - The Apple Orchard }
    \item{ Hundley - Arise My Love }
  \end{enumerate}

  I understand that we have not talked about Gary Bachlund and his art song in this class. Indeed, I'm not sure
  his work is well regarded or popularly sung at all. I found him as a composer by accident, the same way I
  found composer Dan Forrest by accident. Dan Forrest has an SATB setting of the same text - Warm Summer Sun, by Mark
  Twain - titled ``Good Night, Dear Heart", written explicitly in dedication for a child of his friend lost to
  miscarriage. Dan Forrest's setting sounds strikingly similar to Hundley's ``Arise My Love" - and I contend that
  I would prefer Forrest's setting arranged for Solo Voice (singing the Soprano line) and Piano (playing the
  other voice parts) in this recital group instead of the Bachlund song. I found the Bachlund setting on Lieder
  [dot] net while searching if there were other songs with the same text - I only found the Bachlund. It is
  beautiful, and I think the addition of this song and text is, to me, important to hold in discourse with
  ``Arise My Love", hence my addition of a song not on our list to this recital group.

  \clearpage
  \section*{Question 4}
  I don't think I know of a composer for whom I would be less surprised to be smoking out of a cigarette holder.
  He has written \textit{so} much beautiful, expansive music for the piano, all of it heavily inspired by ragtime
  and the jazz-classical fusions of the early-mid 20th century - in some ways, reminiscent of the work of Gulda and
  Kapustin and Gershwin. The song amor, I would say, while not as evidently ragtime, is still firmly placed in this
  intersection of jazz and western classical music in the early 20th century. The simplicity and attractiveness of
  the melody is so hard to not fall in love with - like a lot of his piano works. Should I get to programme
  this piece, I would love to put behind - sort of an ``encore" to - MacDowell's ``From an Old Garden".


  \clearpage
  \section*{Question 5}
  \begin{itemize}
    \item {Laitman}\begin{itemize}
      \item{I was aware of Lori Laitman before this class but, having never read her website, I was very ill-informed
          of her musical career and voice. I am absolutely enamored by her ``informal biography", and quite taken
          by the concept of a composer who ``accidentally" became a composer. This background lends a sort of
        earnest intimacy and levity to her work for me.}
      \item{I also just found out that she completed Ludlow Act 2 AND I MUST FIND RECORDINGS FROM IT.}
    \end{itemize}
    \item {Scott Wheeler}\begin{itemize}
        \item{I have not listened to much of Scott Wheeler, so I didn't have much of an idea of him as a composer. I have
          been, for a while, interested in hearing ``Democracy" - I listened to some selections while reading his website,
        and I have to admit that I am thoroughly intrigued. I do quite like the inclusion of a performance calendar, so I
      can immediately know when and where his music is to be performed. His confidence terrifies me - on his ``contact"
      page, there is a button that IMMEDIATELY TAKES YOU TO EMAIL ``SCOTTWHEELER24@GMAIL.COM". Ignoring the very casual
      naming of his email address - and that it goes to a gmail address and is not forwarding from [user]@scottwheeler.org -
      having an email address be opened to the public rather than having contact setup through a web form of some sort is
        inconceivable for me. }
    \end{itemize}
  \end{itemize}

  \nocite{*}

\printbibliography

\end{document}
