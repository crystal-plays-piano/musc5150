\documentclass{article}

% - imports
\usepackage{enumitem}
\usepackage[utf8]{inputenc}
\usepackage[backend=biber]{biblatex-chicago}
\addbibresource{bibliography.bib}
\usepackage{url}
\usepackage{parskip}
\usepackage{indentfirst}
\usepackage{parallel}

\title{\vspace*{-72pt} Title}
\author{Crystal Mandal}
\date{Last Edited \today}

\begin{document}

  \maketitle

  \section*{Question 1}
    Unfortunately for Graham Johnson, I find very little to like
    in these programme notes. If we are to number and label the
    sentences ("first", "second", ..., "last") in order, of the
    six total sentences only the third, fourth, and fifth sentences
    border on utility as programme notes. In fact, in a
    bizarre stylistic choice that mimics Johnson's own
    description of the text - "sheer insouciant nonsense,
    quasi-surrealist" - the first and last sentences of the full
    note are only barely intelligible. Who does it serve to
    introduce such a delightful and "silly song" - in Johnson's
    own words a "popular hit" - with such densely structured and
    obscure language that I can hardly tell if Johnson himself
    even \textit{likes} this song?

    Within such a labyrinthine programme note is hidden minimal
    context: the publication date, a hint of lyric and musical
    description ("silly song", "words are sheer whimsy"), and
    a rigorous explanation of the narrative. Subtracting these
    brief and illuminating comments, I'm not sure the rest of
    the paragraph fullfils any requirements of an insigtful
    programme note other than reaching a particular word count.

  \clearpage{}
  \section*{Question 2}
    This is a beautiful song that, like much of Poulenc's mélodie,
    moves me Unfortunately very little. Most of my thoughts deal
    with the poem. Pardon the informality, but \textit{ HOW IN THE
    WORLD DOES A COMPOSER SET (is the word ergodic?) A POEM LIKE THIS?}
    I don't know if I think that the music can ever properly depict this
    beautiful layout and the imagery implicated by it. I'm also sure the
    allusion to the cornflower as noted in \textit{A French Song Companion}
    \autocite[375]{french-song-companion} is lost on me. Also interesting
    to note is the subject of the World War One soldier of the original
    poem and the implied subject of World War Two soldier in Poulenc's
    setting (Poulenc's piece is from 1939, the start of WWII). Why is it
    that Poulenc's only musical contribution to the higher (``younger"?)
    ``male" voice is the song of and for a dying young man sent of to war?

  \clearpage{}
  \section*{Question 3}
    I think this is the first set of pieces where I, as a pianist,
    can say that I've heard Poulenc in them. Poulenc's style on the
    piano is very distinct, playful, dramatic, and virtuosic. There's
    also a healthy nostalgia for French Romantic music in his writing,
    and it usually shows in the textures, melodies, and harmonic
    structures in his piano works. Unfortunately, I disagree with
    Kimball's assertion that ``the piano shares equal role with voice
    in any Poulenc song". Many of the Poulenc songs we've so far
    listened to in class have been beautiful, but none of them as
    much as these three have so far treated the piano with the
    respect and experience that, in my opinion, Poulenc treats the
    piano soloist or the orchestral pianist. In this cycle, I'm
    most excited about the second and third pieces. The second
    piece starts with a beautifully ornamented and voiced opening
    with solo piano. The heavy ornamentation and counterpoint
    continue under the voice, with the voice frequently doubling
    the primary melody. The song is quite reminiscent of Schumann's
    \textit{Kreisleriana} Op. 16, specifically the fourth movement.
    The last movement is also quite notable. ``Paganini" refers,
    of course, to the great violinist Niccoló Paganini, but also
    calls into conversation the great body of Romantic Piano work
    that imitates Paganini. The wide leaps, quick figuration, and
    intensely harmonic (with very smooth voice leading, if one
    ignores octaves) motion is all very idiomatic both of Paganini
    and all the work that references his distinct style.

  \clearpage{}
  \section*{Question 4}
    \begin{itemize}
      \item{Biography}\autocite[45-46]{french-song-companion} \begin{itemize}
        \item{Dates: 16 September 1887 – 22 October 1979}
        \item{Composition Dates:}\begin{itemize}
          \item{Cut short by sister's death}
          \item{Lili's death: 1918}
          \item{Last published work of Nadia: 1922, 5 Songs after
            Camille Mauclair}
        \end{itemize}
      \end{itemize}

      \item{Components of Style}\begin{itemize}
        \item{Beautiful, lyrical melodies}
        \item{If Lili is the portrait of the French Radical composer,
          Nadia reflects a more conservative outlook}
        \item{Simple, but not uninteresting piano parts}
        \item{Piano in almost exclusively "accompaniment" parts, made
          interesting texturally by full use of range and an intimate
          understanding of voicing for the instrument}
        \item{Long, moving phrases with lots of functional harmony
          and deceptive or otherwise ``sidestepped" cadences (similar
          in style to Ravel's slower work. compare the "Cantique" and
          Ravel's Piano Concerto in G, 2nd mvt.}
      \end{itemize}
    \end{itemize}




  % 
\vspace*{-72pt}

\begin{Large}
  Song Sheet: F. Schubert's \textit{Der Wanderer} 
\end{Large}


% \begin{enumerate}[label=\Alph*.)]
%   \item{The Chosen Song is: Franz Schubert's \textit{Der Wanderer}, Op. 65, No., D.649 }
% \end{enumerate}

\begin{itemize}[label=]
  \item{\textbf{Title: \textit{Der Wanderer}, Op. 65, No. 2, D. 649}}
  \item{\textbf{Composed: February 1819}}
  \item{\textbf{Composer: Franz Schubert (1797-1828)}}
  \item{\textbf{Poet: Georg Philipp Schmidt ('von Lübeck', 1766-1849) formerly attributed to Zacharias Werner }}\autocite{imslp-649}
\end{itemize}



\section*{General}
This song is one of four composed by Schubert on the same poem. 
The other three are: D.649, D.795, and D.8D.649, D.795, and D.870 \autocite{tapatalk} ; 
of note, D.649 is the most well known (at least, amongst pianists) of 
the bunch, and is - quite famously - the primary subject of the great 
\textit{Fantasy in C Major} (or ``Wanderer Fantasy") of Schubert.

Approximate Performance Time: a little under 4 minutes.

\section*{Melody}

\begin{tabular}{ c | c }
  Melodic Contour    & mostly stepwise or by thirds \\ 
  Tessitura          & spans a Major 9th, from C4 to D5\\
  Vocal Articulation & I'm not sure what this means\\
  Text Illustration  & N/A
\end{tabular}



\section*{Harmony}

\begin{tabular}{ c | c }
  Texture           & Chorale Texture \\ 
  Tonality          & Minimal Modulation \\
                    & Some Chromatic Passages\\
  Text Illustration & N/A\\
\end{tabular}



\section*{Rhythm}

\begin{tabular}{ c | c }
  Rhythmic Pattern & Flowing Chorale in 8th notes \\
                   & few ``marching" interjections\\ 
  Tempo & Andante\\
\end{tabular}



\section*{The Piano Component}

\begin{tabular}{ c | c }
  Preludes/Interludes/Postludes & few solo piano passages\\ 
  Tonality       & Some chromatic passages \\
                 & heavy insistence on a melodic G\#\\
  Use of Motives & there are many references to \\
                 & the opening piano gesture\\
\end{tabular}



\section*{Poem/Text}

\begin{large}
  \textbf{\textit{Der Wanderer} - Georg Philipp Schmidt 'von Lübeck' }
\end{large}

\vspace{10pt}

\begin{Parallel}[v]{0.48\textwidth}{0.48\textwidth}
  \ParallelLText{
    Wie deutlich des Mondes Licht \\
    Zu mir spricht, \\
    Mich beseelend zu der Reise:\\
    "Folge treu dem alten Gleise,\\
    Wähle keine Heimath nicht.\\
    Ew'ge Plage\\
    Bringen sonst die schweren Tage.\\
    Fort zu andern\\
    Sollst du wechseln, sollst du wandern,\\
    Leicht entfliehend jeder Klage."\\

    Sanfte Ebb' und hohe Fluth,\\
    Tief im Muth,\\
    Wandr' ich so im [Dunkel]1 weiter,\\
    Steige muthig, singe heiter,\\
    Und die Welt erscheint mir gut.\\
    Alles reine\\
    Seh' ich mild im Wiederscheine,\\
    Nichts verworren\\
    In des Tages Gluth verdorren:\\
    Froh umgeben, doch alleine.}\

  \ParallelRText{
    How clearly the moon's light \\
    Speaks to me, \\
    Inspiring me to journey; \\
    "Follow truly the ancient path, \\
    Choose no homeland whatsoever. \\
    Otherwise the heavy days bring \\
    Endless troubles ; \\
    Away, to the other \\
    Should you change, should you wander, \\
    Lightly shedding every woe." \\

    Gentle ebb and lofty flood, \\
    Deep in courage, \\
    I wander farther in darkness, \\
    I climb bravely, singing cheerfully, \\
    And the world seems good to me. \\
    All pureness \\
    See I softly in the twilight, \\
    Without confusion \\
    Fading in the day's afterglow: \\
    Surrounded by joy, but alone.} 
\end{Parallel}


\section*{Poet}


I couldn't find much information of the Poet, simply that 
he is known for this poem alone.

\section*{Choice of Text} 
Schubert has multiple songs and other works arranging this
text. Each of his settings occupies a unique soundworld and 
emotional context. This setting, compared to his more/ dramatic 
and popular - at least amongst pianists - D.493, is more calm 
and reticent, lending to a figure of a weary, wise, and well 
travelled \textit{Wanderer}.


\section*{Prosody}
I'm not familiar with much German diction, so it seems very 
normal/characteristically German in pronunciation and stress 
to me.

  % 
\vspace*{-72pt}

\begin{Large}
  Song Sheet: R. Schumann's \textit{Der Contrabandista} 
\end{Large}


% \begin{enumerate}[label=\Alph*.)]
%   \item{The Chosen Song is: Franz Schubert's \textit{Der Wanderer}, Op. 65, No., D.649 }
% \end{enumerate}

\begin{itemize}[label=]
  \item{\textbf{Title: \textit{Der Contrabandiste}, Op. 74 }} 
  \item{\textbf{Composer: Robert Schumann (1810-1856)}}
  \item{\textbf{Poet: Emanuel von Geibel (1815 - 1884) }}\autocite{imslp-74}
\end{itemize}



\section*{General}

This song is from the appendix to R. Schumann's 
\textit{Spanisches Liederspeiel}. According to 
Hyperion records, it was removed from the original 
set due to its relative unimportance to the storyline 
of the overall set.



Approximate Performance Time: a little over 1 minute

\section*{Melody}

\begin{tabular}{ c | c }
  Melodic Contour    & very chordal, lots of jumps\\ 
  Tessitura          & almost 2 octaves, from A2 to G4\\
  Vocal Articulation & I'm not sure what this means\\
  Text Illustration  & N/A
\end{tabular}



\section*{Harmony}

\begin{tabular}{ c | c }
  Texture           & heavily arpeggiated \\ 
  Tonality          & Minimal Modulation \\
  Text Illustration & N/A\\
\end{tabular}



\section*{Rhythm}

\begin{tabular}{ c | c }
  Rhythmic Pattern & steady 8th notes with decoration\\
                   & mismatched duplets and triplets\\
  Tempo            & Schnell (Fast)\\
\end{tabular}



\section*{The Piano Component}

\begin{tabular}{ c | c }
  Preludes/Interludes/Postludes & simple I - V introduction\\ 
  Tonality       & conventional CPP harmony\\
  Use of Motives & N/A\\
\end{tabular}

\clearpage

\section*{Poem/Text}

\begin{large}
  \textbf{\textit{Der Contrabandiste} - Emanuel von Geibel }
\end{large}

\vspace{10pt}

\begin{Parallel}[v]{0.48\textwidth}{0.48\textwidth}
  \ParallelLText{
    Ich bin der Contrabandiste, \\
    Weiß wohl Respekt mir zu schaffen. \\
    Allen zu trotzen, ich weiß es, \\
    Furcht nur, die hab' ich vor keinem. \\
    Drum nur lustig, nur lustig! \\
 \\
    Wer kauft Seide, Tabak! \\
    Ja wahrlich, mein Rößlein ist müde, \\
    Ich eil', ja eile, \\
    Sonst faßt mich noch gar die Runde, \\
    Los geht der Spektakel dann. \\
    Lauf nur zu, mein lustiges Pferdchen, \\
    Ach, mein liebes, gutes Pferdchen, \\
    Weißt ja davon, mich zu tragen! \\
  }\

  \ParallelRText{
    I am the smuggler, \\
    And know well how to inspire respect; \\
    I know how to defy everyone, \\
    and I fear no one. \\
    So let us be merry! \\
    Who shall buy my silk and tobacco? \\
    Tryly, my little horse is tired, \\
    I hurry, yes, hurry, \\
    Otherwise the patrol will catch me, \\
    And then things will go very badly! \\
    Run, my merry horse, \\
    Ah, my dear good steed, \\
    You know well how to carry me!
  }\
\end{Parallel}


\section*{Poet}


I couldn't find much information of the Poet, simply that 
he is known for this poem alone.

\section*{Choice of Text} 
Schubert has multiple songs and other works arranging this
text. Each of his settings occupies a unique soundworld and 
emotional context. This setting, compared to his more/ dramatic 
and popular - at least amongst pianists - D.493, is more calm 
and reticent, lending to a figure of a weary, wise, and well 
travelled \textit{Wanderer}.


\section*{Prosody}
I'm not familiar with much German diction, so it seems very 
normal/characteristically German in pronunciation and stress 
to me.

% \nocite{*}
% \printbibliography
\end{document}
