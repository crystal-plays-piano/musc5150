\documentclass{article}

% - imports
\usepackage{enumitem}
\usepackage[utf8]{inputenc}
\usepackage[backend=biber]{biblatex-chicago}
\addbibresource{bibliography.bib}
\usepackage{url}
\usepackage{parskip}
\usepackage{indentfirst}
\usepackage{parallel}

\title{\vspace*{-72pt} Title}
\author{Crystal Mandal}
\date{Last Edited \today}

\begin{document}

  \maketitle

  \section*{Question 1}
    \begin{enumerate}[label=(\alph*)]
      \item{I'm not sure if there is a response requested in this part.}
      \item{I hear a lot of chromaticism and eccentric tonicisations in the music of
        Charles Griffes, especially in the \textit{Three poems by Fiona MacLeod.} Amusingly,
        in the music of MacDowell, especially in the German setting, there are a couple of
        rhythmic gestures very idiomatic of Schumann (and other German composers) that, given
        the context of MacDowell's other music and the American music that follows, are
        given a distinctly \textit{American} flavour. One Stylistic component that I think is
        interesting to note is that both the American and German Romantics are writing similarly
        sounding music about similar topics (nature and love) but the literary movements and
        traditions of the two languages vary wildly. Where much of the music of Schumann draws
        on the mystical nature of the \textit{Forest} - see \textit{Waldszenen}, \textit{Kinderszenen},
        multiple songs in \textit{Dichterliebe}, and more - it seems that a lot of MacDowell
        instead lean on the metaphor of the \textit{Flower} - see \textit{Woodland Sketches},
        especially \textit{To a Wild Rose}, the First Piano Sonata, and \textit{From an Old
        Garden}.}
      \item{The Immediate difference in piano details between \textit{Olben, wo die Sterne glühen} and \textit{The Bluebell} is
        charming. The weight of the ``German" part is immediately noticeable in comparison to the free and
        fantastic nature of the ``American" part. This is even noticeable in the more similar march-like sections of
        \textit{Woodland Sketches} and the First Piano Sonata: MacDowell's ``American" voice uses wider and more
        open voicings. }
    \end{enumerate}

    \clearpage
  \section*{Question 2}
    The first thing I notice is that Beach's setting, for the pianist, seems to contain less of
    her \textit{bravoura} element than I am accustomed to. On second, third, and further listens,
    however, I am convinced that the bravoura element is moved to the voice. Though the range
    seems less wide than other difficult art songs we've seen in this class, the quick pace,
    long phrases, and - to me, at least - maybe not nonsensical, in entirety, but less-than-sensical poetry.

    I'm less familiar - or rather, entirely unfamiliar - with the music of Ned Rorem, but the texture
    immedately invokes a very different experience with the text. The bright, and earnest Beach setting
    is immedately contrasted with a more inward, intimate piano texture and less-melodic vocal line.
    Rather, it seems that the focus is more on the dialogue - more explicitly the canon - between voice
    and piano. The use of less functional harmony and extended chords also positions Rorem adjacent to
    the music of Debussy and Fauré in my mind.

    \clearpage
  \section*{Question 3}
    As is my personal interest, I would like my programme notes to ask some sort of question to be
    answered in discussion with the text and the song. While helpful, these programme notes are
    primarily biographical and thus, to me, quite dry. Some aspects I might like to see elaborated
    on are:
    \begin{itemize}
      \item{Who were Farwell and Cadman's teachers? Farwell's song sounds much less ``conventional" than
        Cadman's. Why is that? }
      \item{The melodies of both songs, as described in these notes, were collected by an ethnologist: who wrote the texts?}
      \item{What other parts of these songs are reflective of their Native American heritage?}
      \item{Lastly, what aspects of Orientalism and exoticism can be applied to these songs? What is being
        said by the composition of an ``Indian Song" by a white composer and a white author's text? Are
        the Native American voices being respectfully treated by these songs?}
    \end{itemize}

    \clearpage
  \section*{Question 4}
    I think this song is lovely. The melody is mostly consonant to the harmony and easy
    to pick out, with frequent doubling in the piano. The harmonic structure is very simple,
    so it won't distract an inexperienced vocalist. The text is heavily repetitive and
    rhyming, with a very memorable ``ritornello" (if I am allowed to call it that). This
    song reminds me of the Donaudy songs we listened to last week.

    \clearpage
  \section*{Question 5}
    Rabindranath Tagore (1861-1941) is without a doubt, the most widely recognised
    Bengali. In 1913, his poetry collection \textit{Gitanjali} (lit.  Songs of Devotion)
    earned him the first Nobel Prize in Literature awarded to an Asian author. His
    work is noted as profoundly sensitive and fresh by the Nobel Foundation \autocite{nobelFoundation}.
    Much of Tagore's work is noteworthy for being translated from the original Bengali to English
    by Tagore himself, keeping his authorial intent between languages, though many purists will
    assert that ``anyone who knows Tagore's poems in their original Bengali cannot
    feel satisfied with any of the translations" \autocite{sen-tagore}. As of 2025, he is
    most known for \textit{Gitanjali}, his short stories, and his many songs, which include
    the national anthems of both Bangladesh (\textit{Amar Shonar Bangla}, lit. ``My Golden Bengal" )
    and India (\textit{Jana Gana Mana}).

  % 
\vspace*{-72pt}

\begin{Large}
  Song Sheet: R. Schumann's \textit{Der Contrabandista} 
\end{Large}


% \begin{enumerate}[label=\Alph*.)]
%   \item{The Chosen Song is: Franz Schubert's \textit{Der Wanderer}, Op. 65, No., D.649 }
% \end{enumerate}

\begin{itemize}[label=]
  \item{\textbf{Title: \textit{Der Contrabandiste}, Op. 74 }} 
  \item{\textbf{Composer: Robert Schumann (1810-1856)}}
  \item{\textbf{Poet: Emanuel von Geibel (1815 - 1884) }}\autocite{imslp-74}
\end{itemize}



\section*{General}

This song is from the appendix to R. Schumann's 
\textit{Spanisches Liederspeiel}. According to 
Hyperion records, it was removed from the original 
set due to its relative unimportance to the storyline 
of the overall set.



Approximate Performance Time: a little over 1 minute

\section*{Melody}

\begin{tabular}{ c | c }
  Melodic Contour    & very chordal, lots of jumps\\ 
  Tessitura          & almost 2 octaves, from A2 to G4\\
  Vocal Articulation & I'm not sure what this means\\
  Text Illustration  & N/A
\end{tabular}



\section*{Harmony}

\begin{tabular}{ c | c }
  Texture           & heavily arpeggiated \\ 
  Tonality          & Minimal Modulation \\
  Text Illustration & N/A\\
\end{tabular}



\section*{Rhythm}

\begin{tabular}{ c | c }
  Rhythmic Pattern & steady 8th notes with decoration\\
                   & mismatched duplets and triplets\\
  Tempo            & Schnell (Fast)\\
\end{tabular}



\section*{The Piano Component}

\begin{tabular}{ c | c }
  Preludes/Interludes/Postludes & simple I - V introduction\\ 
  Tonality       & conventional CPP harmony\\
  Use of Motives & N/A\\
\end{tabular}

\clearpage

\section*{Poem/Text}

\begin{large}
  \textbf{\textit{Der Contrabandiste} - Emanuel von Geibel }
\end{large}

\vspace{10pt}

\begin{Parallel}[v]{0.48\textwidth}{0.48\textwidth}
  \ParallelLText{
    Ich bin der Contrabandiste, \\
    Weiß wohl Respekt mir zu schaffen. \\
    Allen zu trotzen, ich weiß es, \\
    Furcht nur, die hab' ich vor keinem. \\
    Drum nur lustig, nur lustig! \\
 \\
    Wer kauft Seide, Tabak! \\
    Ja wahrlich, mein Rößlein ist müde, \\
    Ich eil', ja eile, \\
    Sonst faßt mich noch gar die Runde, \\
    Los geht der Spektakel dann. \\
    Lauf nur zu, mein lustiges Pferdchen, \\
    Ach, mein liebes, gutes Pferdchen, \\
    Weißt ja davon, mich zu tragen! \\
  }\

  \ParallelRText{
    I am the smuggler, \\
    And know well how to inspire respect; \\
    I know how to defy everyone, \\
    and I fear no one. \\
    So let us be merry! \\
    Who shall buy my silk and tobacco? \\
    Tryly, my little horse is tired, \\
    I hurry, yes, hurry, \\
    Otherwise the patrol will catch me, \\
    And then things will go very badly! \\
    Run, my merry horse, \\
    Ah, my dear good steed, \\
    You know well how to carry me!
  }\
\end{Parallel}


\section*{Poet}


I couldn't find much information of the Poet, simply that 
he is known for this poem alone.

\section*{Choice of Text} 
Schubert has multiple songs and other works arranging this
text. Each of his settings occupies a unique soundworld and 
emotional context. This setting, compared to his more/ dramatic 
and popular - at least amongst pianists - D.493, is more calm 
and reticent, lending to a figure of a weary, wise, and well 
travelled \textit{Wanderer}.


\section*{Prosody}
I'm not familiar with much German diction, so it seems very 
normal/characteristically German in pronunciation and stress 
to me.

% \nocite{*}
% \printbibliography
\end{document}
