\documentclass{article}

% - imports
\usepackage{enumitem}
% \usepackage[backend=biber]{biblatex-chicago}
% \addbibresource{bibliography.bib}
\usepackage{url}
\usepackage{parskip}
\usepackage{indentfirst}
\usepackage[utf8]{inputenc}
\usepackage{parallel}

\title{\vspace*{-72pt} Title}
\author{Crystal Mandal}
\date{Last Edited \today}

\begin{document}

  \maketitle

  \section*{Question 1}
    \begin{enumerate}[label=(\alph*)]
      \item{}\begin{enumerate}
          \item{Who is Chloris?}
          \item{To whom is Hahn writing all these love songs?}
          \item{I'd like some biographical information. Hahn isn't as
            well known as a composer as say, Franck, Fauré, or Ravel -
            other composers of French song. Where/when was he born? With
            whom did he study? What are some hallmarks of his style? What
            should I be listening for?}
      \end{enumerate}
      \item{The three are quite lovely self-contained love songs all by
        the same Composer. Putting them together as a Recital Set is,
        to me, a natural way to sample Hahn's musical styles. I don't
        have a particular inclination to which of \textit{Si mes vers}
        and \textit{Le printemps} should be first, but I do think that
        the end of \textit{À Chloris} is too intimate and lovely to
        not put at the end of the set.}
    \end{enumerate}

\clearpage{}

  \section*{Question 2}
  Topic Chosen: Why is ``Soupir" Challenging for a singer? Or is it?
  \subsection*{Response}
  First, I'm not sure I've heard this piece before, which is such a shame.
  This should be regarded among Ravel's most beautiful melodies alongside
  the G Major Piano Concerto and his String Quartet and Le Tombeau de
  Couperin. So many pieces of his don't even come close to the brilliance
  of instrumentation and melody and the balance of fantastic and grotesque
  he is so well known for that just \textit{shine} in this piece. And,
  just because of all of this minute detail, though I'm not convinced the
  physicality of this piece is too demanding - I'm not a well trained
  vocalist, so I might miss some things regarding technique - I'm
  quite convinced this must be \textit{impossible} to put together.
  I'm not sure, but does the tuning of the strings as harmonics
  require the vocalist to sing closer to a just intonation? If so,
  the piece would require a very well trained ear to even discern
  \textit{why} singing the ``right" notes still sounds ``off". The
  tempo ``Lent" at around 40bpm is just slow enough that an unaccustomed
  singer might find it difficult not to speed up. Of note, also, is the
  very strict rhythm over a very arhythmic accompaniment. The pitch
  and rhythm of the vocalist are almost entirely unsupported by the
  instrumentation. Overall, having to juggle the tuning, strict rhythms,
  and many little duet moments (with the flutes and violas), as well
  as the long phrasing (and, thus, I'd assume difficult breath control)
  make this, in my entirely uninformed opinion, quite challenging for
  a vocalist.

\clearpage{}

  \section*{Question 3}
    \subsection*{Recital Group}
      Total Length: 9:41
      \begin{enumerate}
        \item{Écoutez la chanson bien douce - Nadia Boulanger (6:03) }
        \item{Ach, die Augen sind es wieder - Nadia Boulanger (2:09) }
        \item{Chanson - Nadia Boulanger (1:29) }
      \end{enumerate}
      The obvious thread is that of the composer. Quite predictably,
      the Recital Group I've selected is comprised entirely of Boulanger
      songs. What strikes me about these pieces is the incredibly earnest
      and transparent nature of her music, as well as the absolute mastery
      of lyric and pianistic writing. The first song implores the audience
      "listen to the sweet song" that is to come. It's a love song, not
      for a particular person or character, but for the art of song itself.
      What follows is a short detour into a surprising second language for
      a French Composer. Another love song, though a little more explicit,
      \textit{Ach, die Augen sind es wieder} is musically and narratively
      compelling, with a touch of regret or other bittersweet
      sentimentality. I also firmly believe that it could be sung an
      octave higher to great success. The declaration of love by a higher,
      more conventional (or, as I like to think about it, more
      cisnormative or "cis-approachable") "woman's voice" to,
      in text, another woman queers the music and the space in which it
      is sung, inviting conversations about voice, position, and
      intentionality - and identity! - in musical performance.
      The last piece serves as a lighter "encore" gesture after two much
      more dramatic songs. \textit{Chanson} has some beautiful melodic
      writing and playfully quick rhythms with lots of interjections
      and "acting" involved.

  % 
\vspace*{-72pt}

\begin{Large}
  Song Sheet: F. Schubert's \textit{Der Wanderer} 
\end{Large}


% \begin{enumerate}[label=\Alph*.)]
%   \item{The Chosen Song is: Franz Schubert's \textit{Der Wanderer}, Op. 65, No., D.649 }
% \end{enumerate}

\begin{itemize}[label=]
  \item{\textbf{Title: \textit{Der Wanderer}, Op. 65, No. 2, D. 649}}
  \item{\textbf{Composed: February 1819}}
  \item{\textbf{Composer: Franz Schubert (1797-1828)}}
  \item{\textbf{Poet: Georg Philipp Schmidt ('von Lübeck', 1766-1849) formerly attributed to Zacharias Werner }}\autocite{imslp-649}
\end{itemize}



\section*{General}
This song is one of four composed by Schubert on the same poem. 
The other three are: D.649, D.795, and D.8D.649, D.795, and D.870 \autocite{tapatalk} ; 
of note, D.649 is the most well known (at least, amongst pianists) of 
the bunch, and is - quite famously - the primary subject of the great 
\textit{Fantasy in C Major} (or ``Wanderer Fantasy") of Schubert.

Approximate Performance Time: a little under 4 minutes.

\section*{Melody}

\begin{tabular}{ c | c }
  Melodic Contour    & mostly stepwise or by thirds \\ 
  Tessitura          & spans a Major 9th, from C4 to D5\\
  Vocal Articulation & I'm not sure what this means\\
  Text Illustration  & N/A
\end{tabular}



\section*{Harmony}

\begin{tabular}{ c | c }
  Texture           & Chorale Texture \\ 
  Tonality          & Minimal Modulation \\
                    & Some Chromatic Passages\\
  Text Illustration & N/A\\
\end{tabular}



\section*{Rhythm}

\begin{tabular}{ c | c }
  Rhythmic Pattern & Flowing Chorale in 8th notes \\
                   & few ``marching" interjections\\ 
  Tempo & Andante\\
\end{tabular}



\section*{The Piano Component}

\begin{tabular}{ c | c }
  Preludes/Interludes/Postludes & few solo piano passages\\ 
  Tonality       & Some chromatic passages \\
                 & heavy insistence on a melodic G\#\\
  Use of Motives & there are many references to \\
                 & the opening piano gesture\\
\end{tabular}



\section*{Poem/Text}

\begin{large}
  \textbf{\textit{Der Wanderer} - Georg Philipp Schmidt 'von Lübeck' }
\end{large}

\vspace{10pt}

\begin{Parallel}[v]{0.48\textwidth}{0.48\textwidth}
  \ParallelLText{
    Wie deutlich des Mondes Licht \\
    Zu mir spricht, \\
    Mich beseelend zu der Reise:\\
    "Folge treu dem alten Gleise,\\
    Wähle keine Heimath nicht.\\
    Ew'ge Plage\\
    Bringen sonst die schweren Tage.\\
    Fort zu andern\\
    Sollst du wechseln, sollst du wandern,\\
    Leicht entfliehend jeder Klage."\\

    Sanfte Ebb' und hohe Fluth,\\
    Tief im Muth,\\
    Wandr' ich so im [Dunkel]1 weiter,\\
    Steige muthig, singe heiter,\\
    Und die Welt erscheint mir gut.\\
    Alles reine\\
    Seh' ich mild im Wiederscheine,\\
    Nichts verworren\\
    In des Tages Gluth verdorren:\\
    Froh umgeben, doch alleine.}\

  \ParallelRText{
    How clearly the moon's light \\
    Speaks to me, \\
    Inspiring me to journey; \\
    "Follow truly the ancient path, \\
    Choose no homeland whatsoever. \\
    Otherwise the heavy days bring \\
    Endless troubles ; \\
    Away, to the other \\
    Should you change, should you wander, \\
    Lightly shedding every woe." \\

    Gentle ebb and lofty flood, \\
    Deep in courage, \\
    I wander farther in darkness, \\
    I climb bravely, singing cheerfully, \\
    And the world seems good to me. \\
    All pureness \\
    See I softly in the twilight, \\
    Without confusion \\
    Fading in the day's afterglow: \\
    Surrounded by joy, but alone.} 
\end{Parallel}


\section*{Poet}


I couldn't find much information of the Poet, simply that 
he is known for this poem alone.

\section*{Choice of Text} 
Schubert has multiple songs and other works arranging this
text. Each of his settings occupies a unique soundworld and 
emotional context. This setting, compared to his more/ dramatic 
and popular - at least amongst pianists - D.493, is more calm 
and reticent, lending to a figure of a weary, wise, and well 
travelled \textit{Wanderer}.


\section*{Prosody}
I'm not familiar with much German diction, so it seems very 
normal/characteristically German in pronunciation and stress 
to me.

  % 
\vspace*{-72pt}

\begin{Large}
  Song Sheet: R. Schumann's \textit{Der Contrabandista} 
\end{Large}


% \begin{enumerate}[label=\Alph*.)]
%   \item{The Chosen Song is: Franz Schubert's \textit{Der Wanderer}, Op. 65, No., D.649 }
% \end{enumerate}

\begin{itemize}[label=]
  \item{\textbf{Title: \textit{Der Contrabandiste}, Op. 74 }} 
  \item{\textbf{Composer: Robert Schumann (1810-1856)}}
  \item{\textbf{Poet: Emanuel von Geibel (1815 - 1884) }}\autocite{imslp-74}
\end{itemize}



\section*{General}

This song is from the appendix to R. Schumann's 
\textit{Spanisches Liederspeiel}. According to 
Hyperion records, it was removed from the original 
set due to its relative unimportance to the storyline 
of the overall set.



Approximate Performance Time: a little over 1 minute

\section*{Melody}

\begin{tabular}{ c | c }
  Melodic Contour    & very chordal, lots of jumps\\ 
  Tessitura          & almost 2 octaves, from A2 to G4\\
  Vocal Articulation & I'm not sure what this means\\
  Text Illustration  & N/A
\end{tabular}



\section*{Harmony}

\begin{tabular}{ c | c }
  Texture           & heavily arpeggiated \\ 
  Tonality          & Minimal Modulation \\
  Text Illustration & N/A\\
\end{tabular}



\section*{Rhythm}

\begin{tabular}{ c | c }
  Rhythmic Pattern & steady 8th notes with decoration\\
                   & mismatched duplets and triplets\\
  Tempo            & Schnell (Fast)\\
\end{tabular}



\section*{The Piano Component}

\begin{tabular}{ c | c }
  Preludes/Interludes/Postludes & simple I - V introduction\\ 
  Tonality       & conventional CPP harmony\\
  Use of Motives & N/A\\
\end{tabular}

\clearpage

\section*{Poem/Text}

\begin{large}
  \textbf{\textit{Der Contrabandiste} - Emanuel von Geibel }
\end{large}

\vspace{10pt}

\begin{Parallel}[v]{0.48\textwidth}{0.48\textwidth}
  \ParallelLText{
    Ich bin der Contrabandiste, \\
    Weiß wohl Respekt mir zu schaffen. \\
    Allen zu trotzen, ich weiß es, \\
    Furcht nur, die hab' ich vor keinem. \\
    Drum nur lustig, nur lustig! \\
 \\
    Wer kauft Seide, Tabak! \\
    Ja wahrlich, mein Rößlein ist müde, \\
    Ich eil', ja eile, \\
    Sonst faßt mich noch gar die Runde, \\
    Los geht der Spektakel dann. \\
    Lauf nur zu, mein lustiges Pferdchen, \\
    Ach, mein liebes, gutes Pferdchen, \\
    Weißt ja davon, mich zu tragen! \\
  }\

  \ParallelRText{
    I am the smuggler, \\
    And know well how to inspire respect; \\
    I know how to defy everyone, \\
    and I fear no one. \\
    So let us be merry! \\
    Who shall buy my silk and tobacco? \\
    Tryly, my little horse is tired, \\
    I hurry, yes, hurry, \\
    Otherwise the patrol will catch me, \\
    And then things will go very badly! \\
    Run, my merry horse, \\
    Ah, my dear good steed, \\
    You know well how to carry me!
  }\
\end{Parallel}


\section*{Poet}


I couldn't find much information of the Poet, simply that 
he is known for this poem alone.

\section*{Choice of Text} 
Schubert has multiple songs and other works arranging this
text. Each of his settings occupies a unique soundworld and 
emotional context. This setting, compared to his more/ dramatic 
and popular - at least amongst pianists - D.493, is more calm 
and reticent, lending to a figure of a weary, wise, and well 
travelled \textit{Wanderer}.


\section*{Prosody}
I'm not familiar with much German diction, so it seems very 
normal/characteristically German in pronunciation and stress 
to me.

% \nocite{*}
% \printbibliography
\end{document}
