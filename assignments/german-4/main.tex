

\documentclass{article}

% - imports
\usepackage{enumitem}
\usepackage[backend=biber]{biblatex-chicago}
\addbibresource{bibliography.bib}
\usepackage{url}
\usepackage{parskip}
\usepackage{indentfirst}

\usepackage{parallel}

\title{\vspace*{-72pt} German 3}
\author{Crystal Mandal}
\date{Last Edited \today}

\begin{document}

  \maketitle

  \section*{Question 1}
  \begin{itemize}
    \item{I am very comfortable with recital planning as a pianist, so 
      many of the concepts are familiar to me. I am quite familiar with 
      the necessity of careful planning regarding the ``difficulty" (or, 
      in my opinion and in the case of a solo pianist, the physical 
      exhaustion) of performing the repertoire. In my last recital, I 
      put a great deal of thought into picking the repertoire to be 
      thematically and structurally cohesive.}
    \item{I did know about but have never really given any thought to the 
      number of songs sung in a recital. The text instructs us on a 
      ``standard" recital programme of five groups with four to six 
      songs each, or a total of almost \textit{thirty songs!} This is, 
      to me, an unthinkably long programme: even if I were to collect, 
      programme, and perform a collection of ``songs" and their 
      counterparts for piano, I couldn't imagine such a long programme. }
    \item{While I disagree with such explicit structuring or templating, 
      I do agree that it can be a great starting point. } 
    \item{I think I will have to do a great deal more of song literature 
      study to draft a recital programme I like.}
    \item{No questions for now, though I am, as evidenced by prior 
      performances, \textit{wholly} ignorant about and 
      usually unprepared for encores.}

  \end{itemize}

\section*{Question 2}

For this recital group, I have picked:
\begin{enumerate}
  \item{Im wunderschönen Monat Mai - R. Schumann Op. 48 No. 1 (~2 mins)}
  \item{Liebestreu - J. Brahms Op. 3 No. 1 (~2 mins)}
  \item{Die Mainacht - J. Brahms Op. 43 No. 2 (~3-4 mins)}
  \item{O Tod, wie bitter bist du - J. Brahms Op. 121 No 3. (~4-5 mins)}
  \item{Widmung - R. Schumann Op. 25 No. 1 (~2 mins)}
  % \item{Die Mainacht - Schubert}
\end{enumerate}
\textit{Total: approximately 13 to 15 minutes in length.}


Both my ignorance surrounding difficulty and style in vocal literature
and my personal insistence on gender-neutral voice casting in solo work
inform my amateurish assertion that these songs are all to be performed
by a single performer. I selected these pieces in particular for the
overall variety in character and texture, along with their general
reticence on love in various forms (and the recurring month of May).
Though I wanted to place \textit{Die Mainacht} and
\textit{Im wunderschönen Monat Mai} following one another, I was too
drawn to the lovely deceptive cadence outlined by the unresolved
$C\#^{7}$ chord at the end of the Schumann song directly into the
E-flat or ``D-sharp" tonality at the beginning of \textit{Liebestreu}.
Another fun relationship is the implicit, turbulent discourse on love
surrounding Robert Schumann, Johannes Brahms, and Clara Schumann. I
am yet undecided if the inclusion of a song by Clara Schumann would
undermine or distract from the dialogue between Robert Schumann and
Johannes Brahms (or if at all, in fact, this dialogue is meaningful or
interesting in any way).

\section*{Question 3}
For this question, I selected the Strauss pair for two very simple and 
insufficient reasons: 
\begin{enumerate}
  \item{I love Richard Strauss: Elektra is my favourite opera.}
  \item{I despise Mahler: he's not especially significant in piano 
    literature, and I keep mixing him up with Wagner - also unimportant 
    in piano literature - who has very good reasons to dislike, even if 
    the Liebestod from \textit{Tristan und Isolde} is among the most 
    beautiful things I've ever heard.}
\end{enumerate}

For the version with piano accompaniment: 
\begin{enumerate}
  \item{The clarity of the piano texture facilitates understanding of 
    Strauss' harmonic choices.}
  \item{The voice is never drowned out by the piano.}
  \item{A single instrumentalist can follow a vocalist in such an 
    intimate fashion that could never be replicated by an entire 
    ensemble. There's a brilliant recording of Christa Ludwig and 
    Leonard Bernstein arguing over tempo: in the end, Ludwig still 
    has to follow the conductor, but with a more intimate piano 
    accompaniment, a vocalist could instead lead the song in a 
    more natural, almost conversational tempo.}
\end{enumerate}

For the version with orchestral accompaniment: 
\begin{enumerate}
  \item{I believe Richard Strauss, unlike many others, is truly 
    capable of touching the sublime with his orchestration. Anything 
    of his with an orchestra involved is something substantial to 
    listen to and appreciate. }
  \item{Here, many inner voices and rhythms that are left out of the 
    piano part to facilitate playing and clarify the accompaniment are 
    given a proper treatment as beautiful additions to wind parts and 
    horn parts (especially, in my opinion, the addition of quicker 
    arpeggiated figures in the oboes). }
  \item{It is quite clear that Strauss is much more familiar with 
    an orchestral idiom than he is at the piano. The music is much 
    more recognizable to me as Strauss, and feels more ``comfortable", 
    compositionally.}
\end{enumerate}

\section*{Question 4}

As Roger Vignoles notes in his insightful commentary as found on Hyperion 
Records \autocite{hyperion-notes}, the piano part of these Mahler songs 
reflects Mahler's own familiarity with the orchestral idiom. Indeed, in 
this selection ``Hans und Grethe", the piano texture recalls the horns 
as depicted in the ritornello of Franz Liszt's fifth Paganini Etude 
or the first theme of Alkan's famed G Sharp Minor etude.

  % 
\vspace*{-72pt}

\begin{Large}
  Song Sheet: F. Schubert's \textit{Der Wanderer} 
\end{Large}


% \begin{enumerate}[label=\Alph*.)]
%   \item{The Chosen Song is: Franz Schubert's \textit{Der Wanderer}, Op. 65, No., D.649 }
% \end{enumerate}

\begin{itemize}[label=]
  \item{\textbf{Title: \textit{Der Wanderer}, Op. 65, No. 2, D. 649}}
  \item{\textbf{Composed: February 1819}}
  \item{\textbf{Composer: Franz Schubert (1797-1828)}}
  \item{\textbf{Poet: Georg Philipp Schmidt ('von Lübeck', 1766-1849) formerly attributed to Zacharias Werner }}\autocite{imslp-649}
\end{itemize}



\section*{General}
This song is one of four composed by Schubert on the same poem. 
The other three are: D.649, D.795, and D.8D.649, D.795, and D.870 \autocite{tapatalk} ; 
of note, D.649 is the most well known (at least, amongst pianists) of 
the bunch, and is - quite famously - the primary subject of the great 
\textit{Fantasy in C Major} (or ``Wanderer Fantasy") of Schubert.

Approximate Performance Time: a little under 4 minutes.

\section*{Melody}

\begin{tabular}{ c | c }
  Melodic Contour    & mostly stepwise or by thirds \\ 
  Tessitura          & spans a Major 9th, from C4 to D5\\
  Vocal Articulation & I'm not sure what this means\\
  Text Illustration  & N/A
\end{tabular}



\section*{Harmony}

\begin{tabular}{ c | c }
  Texture           & Chorale Texture \\ 
  Tonality          & Minimal Modulation \\
                    & Some Chromatic Passages\\
  Text Illustration & N/A\\
\end{tabular}



\section*{Rhythm}

\begin{tabular}{ c | c }
  Rhythmic Pattern & Flowing Chorale in 8th notes \\
                   & few ``marching" interjections\\ 
  Tempo & Andante\\
\end{tabular}



\section*{The Piano Component}

\begin{tabular}{ c | c }
  Preludes/Interludes/Postludes & few solo piano passages\\ 
  Tonality       & Some chromatic passages \\
                 & heavy insistence on a melodic G\#\\
  Use of Motives & there are many references to \\
                 & the opening piano gesture\\
\end{tabular}



\section*{Poem/Text}

\begin{large}
  \textbf{\textit{Der Wanderer} - Georg Philipp Schmidt 'von Lübeck' }
\end{large}

\vspace{10pt}

\begin{Parallel}[v]{0.48\textwidth}{0.48\textwidth}
  \ParallelLText{
    Wie deutlich des Mondes Licht \\
    Zu mir spricht, \\
    Mich beseelend zu der Reise:\\
    "Folge treu dem alten Gleise,\\
    Wähle keine Heimath nicht.\\
    Ew'ge Plage\\
    Bringen sonst die schweren Tage.\\
    Fort zu andern\\
    Sollst du wechseln, sollst du wandern,\\
    Leicht entfliehend jeder Klage."\\

    Sanfte Ebb' und hohe Fluth,\\
    Tief im Muth,\\
    Wandr' ich so im [Dunkel]1 weiter,\\
    Steige muthig, singe heiter,\\
    Und die Welt erscheint mir gut.\\
    Alles reine\\
    Seh' ich mild im Wiederscheine,\\
    Nichts verworren\\
    In des Tages Gluth verdorren:\\
    Froh umgeben, doch alleine.}\

  \ParallelRText{
    How clearly the moon's light \\
    Speaks to me, \\
    Inspiring me to journey; \\
    "Follow truly the ancient path, \\
    Choose no homeland whatsoever. \\
    Otherwise the heavy days bring \\
    Endless troubles ; \\
    Away, to the other \\
    Should you change, should you wander, \\
    Lightly shedding every woe." \\

    Gentle ebb and lofty flood, \\
    Deep in courage, \\
    I wander farther in darkness, \\
    I climb bravely, singing cheerfully, \\
    And the world seems good to me. \\
    All pureness \\
    See I softly in the twilight, \\
    Without confusion \\
    Fading in the day's afterglow: \\
    Surrounded by joy, but alone.} 
\end{Parallel}


\section*{Poet}


I couldn't find much information of the Poet, simply that 
he is known for this poem alone.

\section*{Choice of Text} 
Schubert has multiple songs and other works arranging this
text. Each of his settings occupies a unique soundworld and 
emotional context. This setting, compared to his more/ dramatic 
and popular - at least amongst pianists - D.493, is more calm 
and reticent, lending to a figure of a weary, wise, and well 
travelled \textit{Wanderer}.


\section*{Prosody}
I'm not familiar with much German diction, so it seems very 
normal/characteristically German in pronunciation and stress 
to me.

  % 
\vspace*{-72pt}

\begin{Large}
  Song Sheet: R. Schumann's \textit{Der Contrabandista} 
\end{Large}


% \begin{enumerate}[label=\Alph*.)]
%   \item{The Chosen Song is: Franz Schubert's \textit{Der Wanderer}, Op. 65, No., D.649 }
% \end{enumerate}

\begin{itemize}[label=]
  \item{\textbf{Title: \textit{Der Contrabandiste}, Op. 74 }} 
  \item{\textbf{Composer: Robert Schumann (1810-1856)}}
  \item{\textbf{Poet: Emanuel von Geibel (1815 - 1884) }}\autocite{imslp-74}
\end{itemize}



\section*{General}

This song is from the appendix to R. Schumann's 
\textit{Spanisches Liederspeiel}. According to 
Hyperion records, it was removed from the original 
set due to its relative unimportance to the storyline 
of the overall set.



Approximate Performance Time: a little over 1 minute

\section*{Melody}

\begin{tabular}{ c | c }
  Melodic Contour    & very chordal, lots of jumps\\ 
  Tessitura          & almost 2 octaves, from A2 to G4\\
  Vocal Articulation & I'm not sure what this means\\
  Text Illustration  & N/A
\end{tabular}



\section*{Harmony}

\begin{tabular}{ c | c }
  Texture           & heavily arpeggiated \\ 
  Tonality          & Minimal Modulation \\
  Text Illustration & N/A\\
\end{tabular}



\section*{Rhythm}

\begin{tabular}{ c | c }
  Rhythmic Pattern & steady 8th notes with decoration\\
                   & mismatched duplets and triplets\\
  Tempo            & Schnell (Fast)\\
\end{tabular}



\section*{The Piano Component}

\begin{tabular}{ c | c }
  Preludes/Interludes/Postludes & simple I - V introduction\\ 
  Tonality       & conventional CPP harmony\\
  Use of Motives & N/A\\
\end{tabular}

\clearpage

\section*{Poem/Text}

\begin{large}
  \textbf{\textit{Der Contrabandiste} - Emanuel von Geibel }
\end{large}

\vspace{10pt}

\begin{Parallel}[v]{0.48\textwidth}{0.48\textwidth}
  \ParallelLText{
    Ich bin der Contrabandiste, \\
    Weiß wohl Respekt mir zu schaffen. \\
    Allen zu trotzen, ich weiß es, \\
    Furcht nur, die hab' ich vor keinem. \\
    Drum nur lustig, nur lustig! \\
 \\
    Wer kauft Seide, Tabak! \\
    Ja wahrlich, mein Rößlein ist müde, \\
    Ich eil', ja eile, \\
    Sonst faßt mich noch gar die Runde, \\
    Los geht der Spektakel dann. \\
    Lauf nur zu, mein lustiges Pferdchen, \\
    Ach, mein liebes, gutes Pferdchen, \\
    Weißt ja davon, mich zu tragen! \\
  }\

  \ParallelRText{
    I am the smuggler, \\
    And know well how to inspire respect; \\
    I know how to defy everyone, \\
    and I fear no one. \\
    So let us be merry! \\
    Who shall buy my silk and tobacco? \\
    Tryly, my little horse is tired, \\
    I hurry, yes, hurry, \\
    Otherwise the patrol will catch me, \\
    And then things will go very badly! \\
    Run, my merry horse, \\
    Ah, my dear good steed, \\
    You know well how to carry me!
  }\
\end{Parallel}


\section*{Poet}


I couldn't find much information of the Poet, simply that 
he is known for this poem alone.

\section*{Choice of Text} 
Schubert has multiple songs and other works arranging this
text. Each of his settings occupies a unique soundworld and 
emotional context. This setting, compared to his more/ dramatic 
and popular - at least amongst pianists - D.493, is more calm 
and reticent, lending to a figure of a weary, wise, and well 
travelled \textit{Wanderer}.


\section*{Prosody}
I'm not familiar with much German diction, so it seems very 
normal/characteristically German in pronunciation and stress 
to me.

\nocite{*}
\printbibliography
\end{document}
